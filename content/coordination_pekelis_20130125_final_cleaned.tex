Сочинение

\textbf{Сочинение} --- тип синтаксических отношений, при котором
соединяемые компоненты (конъюнкты) не подчинены один другому и образуют
синтаксическую группу.

В прототипическом случае группа в целом выполняет ту же
семантико-синтаксическую функцию, что и оба ее компонента. А именно, и
конъюнктам, и группе соответствует одинаковая \underline{семантическая
  роль} (для сочинения именных групп) и одинаковый \underline{член
  предложения}.

Так, в сочиненной группе \textit{яблоки и груши} ни один из конъюнктов не
является по отношению к другому вершиной или зависимым и вместе они
образуют составляющую. В составе более широкого синтаксического
контекста (\textit{Я люблю яблоки и груши}) конъюнктам и группе в целом
отвечает одинаковая семантическая роль (в приведенном примере --
Стимула) и одинаковый член предложения (в данном случае прямое
дополнение).

В непрототипическом случае возможно сочинение разных семантических ролей
и разных членов предложения (ср.~\textit{Кто и что говорил?}).

Сочиняться могут единицы почти любого объема --- от клауз (\textit{Нам
  можно}, \textit{а вам нельзя}) до синтаксических групп меньшего объема
(\textit{ни Маша}, \textit{ни Петя}; \textit{вошел и представился}), вершин
синтаксических групп (\textit{то ли над}, \textit{то ли под окном}) и даже
частей слов (\textit{двух- и трехъярусный}).

В русском и многих других языках сочинение оформляется сочинительными
союзами (\textit{и}, \textit{а}, \textit{но}, \textit{или}, \textit{то\ldots то}
и~др.) или не имеет формального показателя (т.н.~бессоюзное сочинение).

Семантически и синтаксически центральной зоной сочинения традиционно
считается конъюнкция, в русском языке оформляемая соединительным союзом
\textit{и}.

Сочинение традиционно противопоставляют \underline{подчинению}
(см.~также \underline{Сочинение и подчинение}).

\tableofcontents

\section{Подходы к
  определению}\label{ux43fux43eux434ux445ux43eux434ux44b-ux43a-ux43eux43fux440ux435ux434ux435ux43bux435ux43dux438ux44e}

Определение сочинения, вынесенное в начало статьи, является
распространенным {[}Dik~1968{]}, {[}Белошапкова~1989{]},
{[}Богуславский~1996{]}~и~др., но не общепринятым.

В формальных синтаксических теориях близкое по сути определение
формулируется в структурных терминах: «Если две клаузы X и Y не вложены
одна в другую и при этом являются непосредственными составляющими
третьей клаузы Z, то отношение между X и Y называется сочинением, а Z
называется сложносочиненной клаузой» {[}Тестелец~2001:~256{]}. В
последние годы, однако, в рамках некоторых синтаксических теорий
возобладало мнение о структурной идентичности сочинения подчинению
(подробнее см.~\underline{п.2.~Структурная интерпретация сочинительной
  конструкции}). Сторонники такой позиции полагают, что из двух конъюнктов
Х и Y один является главным, а другой --- зависимым. Тем самым, сочинение
оказывается структурно неотличимо от подчинения и структурный подход к
определению сочинения автоматически исключается из рассмотрения.

\begin{longtable}[]{@{}lll@{}}
  \toprule
  Что-нибудь & Число1 & Число2 \\
  \midrule
  \endhead
  А          & 1      & 2      \\
  Б          & 2      & 3      \\
  \bottomrule
  \caption{My table}\label{tab:label}
\end{longtable}

\includegraphics[width=1.23667in,height=1.23667in]{media/image1.png}

Помимо синтаксического, существуют семантические и
коммуникативно-прагматические подходы к трактовке сочинения.

Сторонники семантического подхода {[}Munn~2000{]}, {[}Haspelmath~2007{]}
полагают, что основополагающими при сочинении являются семантические
факторы. Так, М.~Хаспельмат называет сочинением «такие синтаксические
конструкции, в которых две или более единицы одного типа объединяются в
одну бóльшую единицу и при этом сохраняют те же семантические отношения
с другими окружающими элементами» {[}Haspelmath~2007:~1{]}.

При сочинении именных групп данное определение, по существу, ставит во
главу угла тождество \underline{семантических ролей}, соответствующих
конъюнктам. Однако такое тождество имеет место только в прототипическом
случае, а в непрототипическом допустимо и сочинение единиц с разными
семантическими ролями (ср. \textit{кто и что говорил}, подробнее
см.~\underline{п.4.3.~Сочинение разных членов предложения}). Между тем
вынесенное в начало статьи синтаксическое определение, акцентирующее
отсутствие между конъюнктами подчинительной связи, не имеет, по крайней
мере в русском языке, бесспорных исключений и поэтому кажется
предпочтительным.

Коммуникативно-прагматические подходы к определению сочинения касаются
только сочинения предложений, или сентенциального сочинения, поскольку
понятие коммуникативной структуры применимо, в первую очередь, к
предложению. Согласно распространенной трактовке, в сложносочиненном
предложении (в отличие от сложноподчиненного) каждой клаузе
соответствует своя иллокутивная сила (о понятии иллокутивной силы
см.~\underline{Модальность}) {[}Cristofaro~1998{]},
{[}Cristofaro~2003{]}, {[}Verstraete~2005{]}. Поэтому сочиняться могут
клаузы с разной иллокутивной силой:

\begin{enumerate}
  \def\labelenumi{(\arabic{enumi})}
  \item
        В принципе я не против кислорода, но почему кислород всегда холодный?
        {[}А.~Солженицын. В круге первом (1968){]}
\end{enumerate}

Все приведенные подходы сходны в двух моментах. Во-первых, они опираются
на интуитивно ощущаемую симметрию сочинения (хотя и трактуют природу
этой симметрии по-разному). Во-вторых, все подходы среди разных аспектов
феномена сочинения --- синтаксического, семантического, коммуникативного
-- выбирают какой-то один в качестве определяющего.

Подходы такого рода можно назвать одноаспектными. Наряду с ними,
существует также многоаспектный подход. При многоаспектном подходе за
основу принимается канонический случай, при котором все аспекты
скоррелированы, и симметрия имеет место на всех уровнях. Если конкретная
конструкция в каком-либо аспекте отходит от канонического типа, это
понижает степень ее «сочинительности». Противопоставление между
сочинением и подчинением, тем самым, оказывается не бинарным, а образует
шкалу, на одном полюсе которой --- безусловно сочинительные конструкции,
на другом --- безусловно подчинительные, и в промежутке --- разного рода
пограничные случаи.

Данный подход, реализованный, в частности, в {[}Подлесская~1992{]}
(«иерархия координативности»), плодотворен, прежде всего при анализе
сочинения в типологическом ракурсе. Существуют языки (например,
алтайские), где сочинение и подчинение часто не разграничены таким
формальным средством, как союз; в этих языках сочинение противостоит
подчинению менее четко и последовательно, чем в русском. В русском же и
типологически близких языках союзы проводят достаточно ясную границу
между сочинением и подчинением, делая возможной и оправданной бинарную
одноаспектную трактовку сочинения\footnote{Бинарный многоаспектный
подход логически тоже возможен, но практически не оправдан. При таком
подходе значительное число конструкций, интуитивно близких к
сочинительным, пришлось бы считать не-сочинительными, т.е., в силу
бинарности подхода, подчинительными из-за отклонений от симметричного
образца в каком-либо аспекте. Так, подчинительными могли бы оказаться
сложные предложения с семантически несимметричным отношением между
клаузами, например, причинно-следственным (\textit{Шел дождь}, \textit{и
  поэтому мы остались дома}). Небинарный многоаспектный подход позволяет
считать такие предложения не подчинительными, а менее сочинительными,
чем семантически симметричные предложения (ср. {[}Подлесская~1992{]}).}.

В самом деле, в русском языке клаузы, вводимые сочинительными и
подчинительными союзами, имеют регулярные формальные отличия
(см.~\underline{Сочинение и подчинение}). Применительно ко всем или
почти всем конструкциям, традиционно считающимся в русском
сочинительными, по-видимому, верно принятое выше определение, основанное
на отсутствии между конъюнктами отношения «вершина~--~зависимое». Таким
образом, этим определением и задается бинарный одноаспектный подход.

Сказанное не означает, однако, что класс сочинительных конструкций
является однородным. Напротив, внутри этого класса имеются как
прототипические сочинительные конструкции, составляющие ядро класса, так
и непрототипические, составляющие периферию. К ядру относятся, прежде
всего, такие сочинительные сочетания, которые не только удовлетворяют
определению сочинения, но и обладают разными проявлениями симметрии,
свойственной сочинению (см. \underline{п.3.~Симметрия сочинения}). К
периферии относятся, в первую очередь, сочетания, удовлетворяющие
\underline{определению сочинения}, но не имеющие тех или иных важных для
сочинения проявлений симметрии (см. \underline{п.4.~Отступления от
  симметричного канона}).

Таким образом, бинарный одноаспектный подход, принятый ниже, с одной
стороны, отражает четкую границу между сочинением и подчинением,
задаваемую союзом в языках европейского стандарта, а с другой --
предусматривает разграничение между каноническим и неканоническим
сочинением.

\section{Структурная интерпретация сочинительной
  конструкции}\label{ux441ux442ux440ux443ux43aux442ux443ux440ux43dux430ux44f-ux438ux43dux442ux435ux440ux43fux440ux435ux442ux430ux446ux438ux44f-ux441ux43eux447ux438ux43dux438ux442ux435ux43bux44cux43dux43eux439-ux43aux43eux43dux441ux442ux440ux443ux43aux446ux438ux438}

Вопрос о структурной интерпретации сочинительной конструкции является
дискуссионным и решается по-разному в рамках разных синтаксических
теорий. Логически возможны два основных варианта структурной
интерпретации:

1)~Так называемая «плоская» структура (англ. flat structure), при
которой оба конъюнкта находятся на одном уровне структурной иерархии,
т.е. ни один из них не вложен в другой: {[}X{]}~(союз)~{[}Y{]}. Такая
структура согласуется с определением сочинения, принятым в начале
статьи, и является традиционной. При данной интерпретации сочинение
оказывается структурно противопоставленным подчинению, имеющему
асимметричную структуру {[}X~{[}(союз)~Y{]}{]}.

2)~Асимметричная структура, при которой один конъюнкт вложен в другой:
{[}X~{[}союз~Y{]}{]}. При данной интерпретации сочинение структурно
тождественно подчинению.

Плоская структура имеет то достоинство, что она отражает присущую
сочинению и разнообразно проявляющуюся симметрию (см.
\underline{п.3.~Симметрия сочинения}).

Плоская структура отражает, кроме того, интуитивно ощущаемую
противопоставленность сочинения подчинению. Недостаток плоской структуры
состоит в том, что она не различает вершину и зависимое и поэтому не
вписывается в теоретические представления об обычном для языка
асимметричном устройстве синтаксической составляющей.

Чтобы устранить указанный недостаток и подогнать структурную трактовку
сочинения под синтаксический «стандарт», в последние десятилетия в
рамках разных теорий (порождающая грамматика, теория «Смысл-Текст»
и~др.) и была предложена асимметричная трактовка сочинения. Такая
трактовка находит поддержку в том, что сочинению, наряду с симметрией,
свойственны и некоторые проявления асимметрии; в частности,
сочинительный союз обычно теснее связан с одним конъюнктом, чем с
другим, просодически и синтаксически (см.~также
\underline{п.4.~Отступления от симметричного канона}).

При асимметричной интерпретации сочинения возможно два варианта анализа
в зависимости от того, какой элемент сочинительного сочетания считается
вершиной --- союз (идея «группы союза» в порождающей грамматике,
{[}Kayne~1994{]}, {[}Johannessen~1998{]}) или линейно первый конъюнкт
(см., например, {[}Velde~2005{]}). Независимо от того, какое решение
принято, общий недостаток асимметричной структуры состоит в том, что
входящие в нее вершина и зависимые отличаются по своим свойствам от
стандартных \underline{вершин и зависимых} (см. Глоссарий). Так, союз,
если считать его вершиной, не придает сочинительному сочетанию особых
дистрибутивных признаков, как это делают стандартные вершины (глагол
определяет дистрибуцию глагольной группы, существительное --- дистрибуцию
именной группы и~т.д.; см. критику «группы союза» в {[}Borsley~2005{]}).

\section{Симметрия
  сочинения}\label{ux441ux438ux43cux43cux435ux442ux440ux438ux44f-ux441ux43eux447ux438ux43dux435ux43dux438ux44f}

Симметрия\footnote{В связи с сочинением чаще используются близкие
  термины \textit{однофункциональность} и \textit{однородность}, имеющие,
  однако, скорее синтаксическую направленность. Термин \textit{симметрия},
  как наиболее общий и нейтральный, в настоящем контексте
  предпочтителен.}, присущая сочинительной конструкции, проявляется на
разных уровнях языка: морфологическом, синтаксическом, семантическом,
коммуникативном. Ниже симметрия сочинения проиллюстрирована в ее
основных аспектах.

\subsection{Морфологический
  аспект}\label{ux43cux43eux440ux444ux43eux43bux43eux433ux438ux447ux435ux441ux43aux438ux439-ux430ux441ux43fux435ux43aux442}

Чтобы сочинение было допустимым, конъюнкты должны иметь определенную
степень морфологического сходства. Эта степень задается списком
ограничений, во многом специфичным для каждого языка. Для русского языка
такой список включает, в частности, следующие ограничения
{[}Кодзасов,~Саввина~1987{]}, {[}Санников~2008{]}:

\begin{itemize}
  \item
        ограничение на сочинение синонимичных грамматических вариантов
        (см.~\underline{п.3.1.1});
  \item
        ограничение на сочинение глагольных форм, различающихся граммемами
        (см.~\underline{п.3.1.2});
  \item
        ограничение на сочинение форм сравнительной степени с положительной
        (см.~\underline{п.3.1.3}).
\end{itemize}

\subsubsection{Ограничение на сочинение синонимичных грамматических
  вариантов}\label{ux43eux433ux440ux430ux43dux438ux447ux435ux43dux438ux435-ux43dux430-ux441ux43eux447ux438ux43dux435ux43dux438ux435-ux441ux438ux43dux43eux43dux438ux43cux438ux447ux43dux44bux445-ux433ux440ux430ux43cux43cux430ux442ux438ux447ux435ux441ux43aux438ux445-ux432ux430ux440ux438ux430ux43dux442ux43eux432}

В позициях, в которых \textsc{морфосинтаксическая} форма конъюнктов
предопределена грамматически, --- т.е., прежде всего, в управляемых
позициях (см.~\underline{Управление}) --- подлежащего\footnote{Именительный
  падеж подлежащего не во всех теориях признается управляемой формой,
  см.~\underline{Падеж}.}, дополнений, именной части сказуемого и~под.
-- затруднено сочинение синонимичных грамматических вариантов:
\textsuperscript{?}\textit{курить и чтение лежа вредно};
\textsuperscript{?}\textit{начальник запретил курить и распитие спиртного
  в вагонах}; \textsuperscript{?}\textit{я не встретил ни Сережу}, \textit{ни
  Вани}; \textsuperscript{?}\textit{он был ленивым и обжора}; \textit{*две
  злые и сварливых собаки}. Исключения встречаются, но носят, как правило,
разговорный характер:

\begin{enumerate}
  \def\labelenumi{(\arabic{enumi})}
  \setcounter{enumi}{1}
  \item
        Мой друг ― доктор из Италии по Интернету посоветовал \textit{мазь} и
        \textit{попить антибиотик}. {[}коллективный. Хватит губить детей!
        (2011){]}
  \item
        Хочется, потому что туда обещали \textit{дрова} \textit{и табаку}.
        {[}Д.~Быков. Орфография (2002){]}
\end{enumerate}

Данное ограничение касается в первую очередь (хотя и не исключительно,
ср.~примеры выше) случаев, когда конъюнкты занимают позицию актанта. Для
сравнения, в позиции обстоятельства подобного ограничения нет (впрочем,
поскольку форма обстоятельства грамматически не предопределена,
применительно к ней и нет смысла говорить о «синонимичных грамматических
вариантах»). Ср.~разное морфосинтаксическое оформление сочиненных
обстоятельств места:

\begin{enumerate}
  \def\labelenumi{(\arabic{enumi})}
  \setcounter{enumi}{3}
  \item
        \textless\ldots\textgreater материал обивки \textit{сзади и по бокам}
        несколько простоват. {[}«За рулем» (2003){]}
\end{enumerate}

Строго говоря, указанное ограничение является не сугубо морфологическим,
а находится на стыке морфологии и синтаксиса: о собственно
морфологической симметрии имеет смысл говорить лишь тогда, когда
сочиняются составляющие одинаковых категорий (ср.
\textsuperscript{?}\textit{не встретил ни Сережу}, \textit{ни Вани}).
Вопросы же, касающиеся сочинения составляющих разных категорий
(\textsuperscript{?}\textit{запретил курить и распитие спиртного в
  вагонах}), уместно скорее отнести к синтаксическому аспекту симметрии
сочинения (см.~об этом \underline{п.3.2.~Синтаксический аспект}). Вместе
с тем, перечисленные запреты представляют собой явления одного порядка
(что и заставляет, пусть и ценой некоторого огрубления, объединить их
под общей рубрикой): затрудненность разного грамматического оформления
конъюнктов в случае, когда это оформление грамматически задано.

\subsubsection{Ограничение на сочинение глагольных форм, различающихся
  граммемами времени и
  вида}\label{ux43eux433ux440ux430ux43dux438ux447ux435ux43dux438ux435-ux43dux430-ux441ux43eux447ux438ux43dux435ux43dux438ux435-ux433ux43bux430ux433ux43eux43bux44cux43dux44bux445-ux444ux43eux440ux43c-ux440ux430ux437ux43bux438ux447ux430ux44eux449ux438ux445ux441ux44f-ux433ux440ux430ux43cux43cux435ux43cux430ux43cux438-ux432ux440ux435ux43cux435ux43dux438-ux438-ux432ux438ux434ux430}

Затруднено сочинение видо-временных форм, семантически не мотивированно
различающихся граммемами времени и вида. Так, сомнительно сочинение
формы будущего времени с формой настоящего в значении «будущего
запланированного» (см.~\underline{Настоящее~время}), поскольку такие
формы близки по видо-временной семантике, но граммемами времени и вида,
тем не менее, различаются. Ср.: \textsuperscript{?}\textit{Завтра я приеду
  и подписываю акт} \textless{}\textit{*приезжаю и подпишу}\textgreater, при
допустимом:

\begin{enumerate}
  \def\labelenumi{(\arabic{enumi})}
  \setcounter{enumi}{4}
  \item
        Завтра я \textit{приеду} \textit{и} \textit{подпишу} акт. {[}В.~Дудинцев. Не
        хлебом единым (1956){]}
\end{enumerate}

При наличии семантической мотивации для различия видо-временных граммем,
такое различие свободно допустимо. Так, форма презенса СВ (`будущее
время') может сочиняться с формой презенса НСВ (`настоящее время') для
выражения следующего смысла: `после того, как было совершено первое
действие, началось второе'. Ср.:

\begin{enumerate}
  \def\labelenumi{(\arabic{enumi})}
  \setcounter{enumi}{5}
  \item
        От людей завесят подшалком, чтоб люди не глядели. Косы \textit{заплетут}
        и \textit{поют} "Трубоньку". {[}«Народное творчество» (2004){]}
\end{enumerate}

По-видимому, различие в приемлемости между сочетаниями
\textsuperscript{?}\textit{приеду и подписываю акт} и *\textit{приезжаю и
  подпишу акт}, рассмотренными выше, связано именно с тем, что только в
первом случае можно с натяжкой усмотреть некоторую семантическую
мотивацию для различия граммем: `после того как приеду, подпишу'.

Это ограничение сходно с ограничением на сочинение синонимических
грамматических вариантов (см.~\underline{п.3.1.1}) в том, что оба они
запрещают разнобой в морфологическом оформлении конъюнктов при близости
их грамматической семантики.

\subsubsection{Ограничение на сочинение форм сравнительной степени с
  формами положительной
  степени}\label{ux43eux433ux440ux430ux43dux438ux447ux435ux43dux438ux435-ux43dux430-ux441ux43eux447ux438ux43dux435ux43dux438ux435-ux444ux43eux440ux43c-ux441ux440ux430ux432ux43dux438ux442ux435ux43bux44cux43dux43eux439-ux441ux442ux435ux43fux435ux43dux438-ux441-ux444ux43eux440ux43cux430ux43cux438-ux43fux43eux43bux43eux436ux438ux442ux435ux43bux44cux43dux43eux439-ux441ux442ux435ux43fux435ux43dux438}

Затруднено сочинение сравнительной степени прилагательных и наречий с
положительной: \textsuperscript{?}\textit{На сей раз мы работали долго и
  внимательней}. Данное ограничение касается преимущественно синтетических
несклоняемых степеней сравнения; аналитические формы сравнения и
склоняемые синтетические (\textit{бóльший}, \textit{мéньший} и~под.)
допускают сочетание с положительной формой --- по-видимому, в силу того,
что морфологическое различие между конъюнктами в этих случаях менее
выраженно, чем при сочинении положительной формы с несклоняемой
синтетической. Ср.:

\begin{enumerate}
  \def\labelenumi{(\arabic{enumi})}
  \setcounter{enumi}{6}
  \item
        Как правило, отношение аудитории к таким фильмам, как «Между стен»
        (\textit{оригинальное} и \textit{более точное} название), всегда
        уважительное, но без фанатизма. {[}коллективный. Класс -\/- Франция
        (2008-2011){]}
  \item
        Одел свой \textit{единственный} и \textit{лучший} костюм. {[}В.~Сидур.
        Памятник современному состоянию (1973-1974){]}
\end{enumerate}

Кроме того, и для несклоняемых синтетических форм ограничение снимается,
если при одном из конъюнктов имеется зависимое: \textit{На сей раз мы
  работали долго и внимательней, чем накануне}. Ср.~также:

\begin{enumerate}
  \def\labelenumi{(\arabic{enumi})}
  \setcounter{enumi}{8}
  \item
        Ты знаешь, он такой \textit{мягкий} и \textit{легче}, чем все прошлые!
        {[}В.~Высоцкий. Как-то так все вышло... (1969-1970){]}
\end{enumerate}

Согласно {[}Кодзасов,~Саввина~1987:~152--153{]}, требование
морфологического сходства не касается морфологических характеристик,
мотивированных семантически (ограничение~2, см.~\underline{п.3.1.2}),
таким образом, --- частное следствие данного обобщения; ограничение~3,
см.~\underline{п.3.1.3}), наоборот, --- исключение из него, поскольку
выбор степени сравнения, очевидно, семантически мотивирован). Так,
сочинению в общем случае не препятствуют разные значения категории числа
у конъюнктов-существительных или категории времени и вида у
конъюнктов-глаголов, ср.:~\textit{Ваня и его сестры}; \textit{лег и спит};
\textit{жил и будет жить}. Ср. также примеры типа \textit{передний и боковые
  ряды}, \textit{белая и цветные рубашки} (см.~о них~\underline{п.6.3.3}):
наличием семантической мотивации разрешается несовпадение у сочиненных
прилагательных граммем числа --- несмотря на то, что прототипически
прилагательное получает граммему числа не семантически, а на основе
синтаксического механизма согласования.

\subsection{Синтаксический
  аспект}\label{ux441ux438ux43dux442ux430ux43aux441ux438ux447ux435ux441ux43aux438ux439-ux430ux441ux43fux435ux43aux442}

О синтаксической симметрии сочинительной конструкции можно говорить
сразу в нескольких смыслах.

Во-первых, согласно данному выше определению, конъюнкты синтаксически
равноправны в том, что ни один из них не поглощает и не распространяет
другой, т.е. не является хозяином или зависимым; они вместе и на равных
правах образуют бóльшую единицу того же типа.

Во-вторых, оба члена сочинительной конструкции, как правило,
представляют собой составляющие. Так, при сочинении, в отличие от
подчинения, допустима замена обоих конъюнктов на местоимения, что
является свойством составляющей (разумеется, замена допустима только в
позициях, в принципе допускающих использование местоимения, т.е., прежде
всего, при сочинении именных групп). Ср.:

\begin{enumerate}
  \def\labelenumi{(\arabic{enumi})}
  \setcounter{enumi}{9}
  \item
        а.~{[}{[}старший сын (он){]} и {[}моя сестра (она){]} --- сочинение
\end{enumerate}

б.~{[}старший сын (*он) {[}моей сестры (ее){]}{]} --- подчинение

О некоторых случаях сочинения единиц, не являющихся каноническими
составляющими, см. \underline{п.6.2.~Сочинение и эллипсис}.

Во-третьих, в наиболее типичном случае сочиняются составляющие
одинаковых категорий, например, именная группа с именной группой,
глагольная с глагольной и~т.д., ср.:

\begin{enumerate}
  \def\labelenumi{(\arabic{enumi})}
  \setcounter{enumi}{10}
  \item
        Рад, что по всем обсуждавшимся вопросам у нас подтвердились
        {[}понимание{]}ИГ и {[}стремление действовать сообща{]}ИГ.
        {[}С.~В.~Лавров. Встреча министров иностранных дел прикаспийских
        государств. Выступление на пресс-конференции по итогам встречи
        (2004){]}
  \item
        Он {[}фыркнул{]}ГГ и {[}ушёл к себе в комнату{]}ГГ. {[}«Даша»
        (2004){]}
\end{enumerate}

Это правило не является, тем не менее, очень строгим, ср.~пример (14)
ниже, подробнее об отступлениях от него см.~\underline{п.4.2}.

В-четвертых, конъюнкты обычно выполняют в предложении одинаковую
синтаксическую функцию, или, иначе, выступают в качестве одинаковых
членов предложения. Ср. сочинение прямых дополнений и обстоятельств,
соответственно:

\begin{enumerate}
  \def\labelenumi{(\arabic{enumi})}
  \setcounter{enumi}{12}
  \item
        Мелко порубить белки, лук, каперсы, анчоусы и травы. {[}Рецепты
        национальных кухонь: Германия (2000-2005){]}
  \item
        И дома, и на работе, и с другом своим ― одинок. {[}В.~Гроссман. Жизнь
        и судьба (1960){]}
\end{enumerate}

Об отдельных случаях сочинения разных членов предложения
см.~\underline{п.4.3}.

Наконец, в-пятых, синтаксическая симметрия проявляется в целом ряде
частных синтаксических особенностей сочинения. Проявлением симметрии
можно считать так называемое «ограничение на сочиненную структуру»,
согласно которому при сочинении запрещены любые синтаксические операции,
применяемые асимметрично --- только к одному конъюнкту сочинительной
конструкции. Ср.~невозможность вопросительного выноса, затрагивающего
лишь один из конъюнктов: *\textit{Кто и Петя пришли?} Подробнее об
«ограничении на сочиненную структуру» см. \underline{п.6.1.~Основные
  синтаксические свойства сочинительной конструкции}.

\subsection{Семантический
  аспект}\label{ux441ux435ux43cux430ux43dux442ux438ux447ux435ux441ux43aux438ux439-ux430ux441ux43fux435ux43aux442}

Семантическая симметрия, как и синтаксическая, проявляется по-разному.

Во-первых, конъюнкты должны иметь определенный уровень семантического
сходства. Поскольку измерить такой уровень затруднительно,
исследователи, как правило, ограничиваются указанием (заведомо
неполного) перечня случаев, когда сочинение невозможно из-за слишком
сильной семантической рассогласованности конъюнктов. Так, в
{[}Кодзасов,~Саввина~1987:~165--167{]}, {[}Санников~2008:~142--146{]}
отмечены следующие случаи.

\subsubsection{Ограничение на сочинение семантически противопоставленных
  единиц}\label{ux43eux433ux440ux430ux43dux438ux447ux435ux43dux438ux435-ux43dux430-ux441ux43eux447ux438ux43dux435ux43dux438ux435-ux441ux435ux43cux430ux43dux442ux438ux447ux435ux441ux43aux438-ux43fux440ux43eux442ux438ux432ux43eux43fux43eux441ux442ux430ux432ux43bux435ux43dux43dux44bux445-ux435ux434ux438ux43dux438ux446}

Затруднено сочинение единиц, противопоставленных по некоторому признаку:
абстрактного и конкретного понятий, имен постоянного и временного
признака и т.п. Так, в сочетании \textit{злой и раздраженный} признак
`злой' естественнее понимается как временный, ввиду его сочинения с
другим временным признаком; а в сочетании \textit{злой и жестокий} тот же
признак скорее интерпретируется как постоянный, поскольку сочиняется с
постоянным признаком. Ср.

\begin{enumerate}
  \def\labelenumi{(\arabic{enumi})}
  \setcounter{enumi}{14}
  \item
        Крутов сидит там один, \textit{злой и раздраженный}, и ждет почту,
        которой нет. {[}Т.~Устинова. Подруга особого назначения (2003){]}
\end{enumerate}

vs.:

\begin{enumerate}
  \def\labelenumi{(\arabic{enumi})}
  \setcounter{enumi}{15}
  \item
        Тем самым я буду доказывать взрослым, что я не забыл обиды, горя,
        пусть знают каждый раз, какие они плохие, \textit{злые и жестокие} люди.
        {[}В.~А.~Солоухин. Смех за левым плечом (1989){]}
\end{enumerate}

Данный запрет не является абсолютным и может нарушаться под влиянием
некоторых факторов. Так, сочинение временного признака с постоянным
допустимо, например, в случае, когда признаки связаны
причинно-следственной связью {[}Санников~2008:~143{]}, ср.~\textit{Он
  молод и погорячился} vs. *\textit{он умен и погорячился}.

\subsubsection{Требование тождества семантических связей конъюнктов с их
  общим зависимым или
  вершиной}\label{ux442ux440ux435ux431ux43eux432ux430ux43dux438ux435-ux442ux43eux436ux434ux435ux441ux442ux432ux430-ux441ux435ux43cux430ux43dux442ux438ux447ux435ux441ux43aux438ux445-ux441ux432ux44fux437ux435ux439-ux43aux43eux43dux44aux44eux43dux43aux442ux43eux432-ux441-ux438ux445-ux43eux431ux449ux438ux43c-ux437ux430ux432ux438ux441ux438ux43cux44bux43c-ux438ux43bux438-ux432ux435ux440ux448ux438ux43dux43eux439}

При сочинении желательно тождество семантических связей конъюнктов с их
общим зависимым или хозяином: общий член не должен выступать по
отношению к конъюнктам в разных значениях. Так, предложения \textit{Шел
  дождь и два студента}, \textit{У него красивые волосы и жена} допустимы
только в качестве каламбура. Как видно из последнего примера, критерии
разграничения значений могут быть весьма тонкими.

Проявлением данного ограничения служит тот, упоминавшийся выше, факт,
что при сочинении именных групп они выступают, как правило, в одинаковой
семантической роли (подробнее см.~\underline{п.1.~Подходы к
  определению}). При этом затруднено сочинение не только таких
семантических ролей, различие между которыми в языке выражено
грамматически --- разными падежами и членами предложения --- но и таких
ролей, различие между которыми в грамматике непосредственно не отражено
(см.~\underline{Семантические роли}). Так, с трудом сочиняются
грамматически одинаково оформленные Инструмент со Средством
(\textsuperscript{?}\textit{писать пером и чернилами}) или Источник с
Бенефактивом (\textsuperscript{?}\textit{заработал на производстве и на
  покупателях}) {[}Тестелец~2001:~211{]}.

Данный запрет, как и предыдущий (см.~\underline{п.3.3.1}), может
преодолеваться, в частности, контекстом. Ср.~пример сочинения разных
семантических ролей (подробнее см.~\underline{п.4.3}):

\begin{enumerate}
  \def\labelenumi{(\arabic{enumi})}
  \setcounter{enumi}{16}
  \item
        Считаю себя вправе писать вам карандашом, в постели и самое домашнее
        письмо (А.~Блок; пример заимствован из {[}Апресян 2010:~231{]})
\end{enumerate}

\subsubsection{Ограничение на сочинение конъюнктов с совпадающими
  денотатами}\label{ux43eux433ux440ux430ux43dux438ux447ux435ux43dux438ux435-ux43dux430-ux441ux43eux447ux438ux43dux435ux43dux438ux435-ux43aux43eux43dux44aux44eux43dux43aux442ux43eux432-ux441-ux441ux43eux432ux43fux430ux434ux430ux44eux449ux438ux43cux438-ux434ux435ux43dux43eux442ux430ux442ux430ux43cux438}

Указанные выше запреты (см.~\underline{п.3.3.1}, \underline{п.3.3.2})
отражают семантическую симметрию сочинения постольку, поскольку
ограничивают допустимую степень семантической рассогласованности
конъюнктов. Однако при сочинении действует и иной, в некотором смысле
противоположный, тип семантического запрета --- ограничивающий степень
допустимого сходства конъюнктов (ср.~невозможность сочинения одинаковых
словоформ: *\textit{стол и стол}). Этот тип запрета отражает уже не
симметрию сочинения, а общеязыковой принцип информативности, согласно
которому высказывание должно содержать новую информацию. Тем не менее,
традиционно соответствующую серию запретов упоминают в одном списке с
запретами \underline{3.3.1} и \underline{3.3.2} --- поскольку все они
объединены тем, что накладывают ограничения на семантику конъюнктов при
сочинении. Речь идет, прежде всего, о следующем общем ограничении,
включающем ряд частных.

1.~Затруднено сочинение конъюнктов, денотаты которых полностью или
частично совпадают. Сюда относятся следующие запреты.

А)~Запрет на сочинение совпадающих словоформ, ср. *\textit{бегемот и
  бегемот}. Показательно, что исключением служат личные местоимения,
употребленные дейктически:

\begin{enumerate}
  \def\labelenumi{(\arabic{enumi})}
  \setcounter{enumi}{17}
  \item
        Пока они собирались да шептались между собой, осмотрел хозяйским
        глазом левую половину вагона, ткнул пальцем в меня и Шабана: «\textit{Ты
          и ты!}». {[}А.~Приставкин. Вагончик мой дальний (2005){]}
\end{enumerate}

Это свидетельствует о том, что запрет касается не лексического, а именно
денотативного совпадения конъюнктов.

Б)~Запрет на сочинение синонимичных существительных, в первую очередь с
предметным значением, ср.~*\textit{бегемот и}
\textless{}\textit{или}\textgreater{} \textit{гиппопотам} (исключение
составляет \textit{или} в пояснительном значении, ср.~\textit{бегемот},
\textit{или гиппопотам}). Синонимичные словоформы других частей речи
сочиняются также ограниченно, но более свободно, чем существительные, --
вероятно, потому, что только в зоне предметной номинации возможна четкая
идентификация денотата. Ср.:

\begin{enumerate}
  \def\labelenumi{(\arabic{enumi})}
  \setcounter{enumi}{18}
  \item
        \textit{Очень и чрезвычайно} вам благодарен, Григорий Иваныч!
        {[}А.~П.~Чехов. Неприятность (1888){]}
  \item
        А человек \textit{способный и талантливый} гораздо меньше озабочен этим
        самым будущим \textless\ldots\textgreater. {[}Б.~Немцов. Провинциал в
        Москве (1999){]}
\end{enumerate}

2.~Запрет на сочинение частного понятия и соответствующего ему общего:
*\textit{На поляне мы нашли малину и ягоды}; *\textit{Ловля судака и рыбы
  запрещена}; \textsuperscript{\#}\textit{Они грабили, убивали и совершали
  преступления}.

Последнее ограничение может быть снято (правда, только на «поверхностном
уровне» --- см. ниже) одним из двух способов. Во-первых, наличием при
имени общего понятия определений типа \textit{другой}, \textit{прочий}
и~т.п., позволяющих исключить из множества денотатов этого имени денотат
имени частного понятия, избежав тем самым совпадения денотатов:

\begin{enumerate}
  \def\labelenumi{(\arabic{enumi})}
  \setcounter{enumi}{20}
  \item
        Кроме того, цветки липы активно посещают \textit{шмели и другие
          насекомые}. {[}«Лесное хозяйство» (2003){]}
\end{enumerate}

Во-вторых, ограничение снимается наличием при одном из конъюнктов
зависимого, эксплицирующего факт совпадения денотатов:

\begin{enumerate}
  \def\labelenumi{(\arabic{enumi})}
  \setcounter{enumi}{21}
  \item
        Я изучением жизни \textit{птиц}, \textit{и особенно ворон}, увлекаюсь.
        {[}В.~Черкасов. Черный ящик (2000){]}
\end{enumerate}

В последнем случае, однако, конструкция по ряду свойств отклоняется от
канонической сочинительной, поэтому в русистике сочетания типа (22)
трактуются не как сочинительные, а как обособленные
{[}Грамматика~1980(2):~181{]}. Так, в (22) допустимо опущение союза,
тогда как в стандартной конструкции с союзом \textit{и}, где союз
соединяет семантически однотипные объекты, такое опущение невозможно:

\begin{enumerate}
  \def\labelenumi{(\arabic{enumi})}
  \setcounter{enumi}{22}
  \item
        а.~Я изучением жизни птиц, особенно ворон, увлекаюсь.
\end{enumerate}

б.~Я увлекаюсь изучением жизни галок, синиц, воробьев *(и) особенно
ворон.

Таким образом, можно заключить, что запрет на сочинение частного и
общего понятий все-таки не может быть по-настоящему обойден: приведенные
выше способы снятия этого запрета или убирают основания для того, чтобы
запрет вступил в силу, или выводят конструкцию за рамки
сочинения\footnote{Конструкция типа \textit{Я изучением жизни птиц},
  \textit{и особенно ворон}, \textit{увлекаюсь} вызывает целый ряд вопросов:
  каков исчерпывающий перечень свойств, отличающих ее от канонического
  сочинения? каков грамматический статус слова \textit{и} в данной
  конструкции? каков статус всей конструкции с точки зрения оппозиции
  сочинение vs. подчинение? Эти вопросы требуют дальнейшего изучения.}.
По-видимому, верно и следующее общее утверждение: запреты,
ограничивающие семантическое сходство конъюнктов, сильнее запретов,
ограничивающих их семантическую рассогласованность, --- последние, как
показано выше (см.~\underline{п.3.3.1} и \underline{п.3.3.2.}),
нарушаться могут.

\subsubsection{Обратимость
  конъюнктов}\label{ux43eux431ux440ux430ux442ux438ux43cux43eux441ux442ux44c-ux43aux43eux43dux44aux44eux43dux43aux442ux43eux432}

Помимо семантического сходства, второе проявление семантической
симметрии при сочинении состоит в том, что сочиненные члены часто бывают
обратимы, т.е. могут меняться местами при сохранении на прежнем месте
союза и без существенных изменений смысла: \textit{Им можно}, \textit{а вам
  нельзя} ≈ \textit{Вам нельзя}, \textit{а им можно}. Ср.~также:

\begin{enumerate}
  \def\labelenumi{(\arabic{enumi})}
  \setcounter{enumi}{23}
  \item
        Среди неопределившихся остаются \textit{Пакистан}, \textit{Чили},
        \textit{Мексика и Ангола}. {[}«Известия» (2003){]} ≈ \textit{Ангола},
        \textit{Мексика}, \textit{Чили и Пакистан}
\end{enumerate}

Обратимость при сочинении связана с семантикой союза --- с тем, что
сочинительный союз часто (несоизмеримо чаще, чем подчинительный)
выражает семантически симметричное отношение, т.е. такое, при котором
конъюнкт А семантически относится к конъюнкту В так же, как В относится
к А, см.~статьи \underline{Союз} и \underline{Сочинение и подчинение}.
Так, семантически симметрично отношение, выражаемое союзом \textit{и} в
перечислительном значении или союзом \textit{или} (поэтому \textit{Маша и}
\textless{}\textit{или}\textgreater{} \textit{Петя} ≈ \textit{Петя и}
\textless{}\textit{или}\textgreater{} \textit{Маша}). Но, к примеру,
причинно-следственное отношение, выражаемое подчинительным союзом
\textit{потому что}, не является семантически симметричным: один член
этого отношения выступает как причина, другой --- как следствие (поэтому
\textit{ушел}, \textit{потому что обиделся} ≠ \textit{обиделся}, \textit{потому
  что ушел}). Ввиду сказанного, обратимость естественно считать
проявлением именно семантической симметрии (хотя формально возможность
перестановки конъюнктов можно было бы счесть и синтаксическим
свойством).

О случаях, когда обратимость при сочинении отсутствует, т.е. порядок
конъюнктов фиксирован, см. \underline{4.1.~Необратимость при сочинении}.

\subsection{Коммуникативный
  аспект}\label{ux43aux43eux43cux43cux443ux43dux438ux43aux430ux442ux438ux432ux43dux44bux439-ux430ux441ux43fux435ux43aux442}

Коммуникативная симметрия сочинения проявляется, во-первых, в том, что
обе сочиненные клаузы имеют иллокутивную силу --- в отличие от
подчинения, при котором одна из клауз может быть лишена иллокутивной
силы (подробнее см.~\underline{п.1.~Подходы к определению} и статью
\underline{Сочинение и подчинение}).

Во-вторых, элементы сочиненной конструкции обычно не могут быть
коммуникативно противопоставлены в каком-либо отношении. Так, затруднено
контрастивное выделение на сочиненной группе, затрагивающее только один
конъюнкт. Cр.~сомнительность контрастивного акцента на одном из
конъюнктов, при котором второй конъюнкт остается вне сферы действия
контраста: \textsuperscript{?}\textit{Маша и Ваня пришли!}
\textless{}\textit{а не Маша и Петя}\textgreater. Контраст, выраженный не
интонационно, а лексически, и вовсе неприемлем: *\textit{Маша и}
\textless{}\textit{или}\textgreater{} \textit{именно Ваня пришли}.

С коммуникативной симметрией при сочинении связан и следующий запрет:
именная группа, входящая в сочинительную конструкцию, не может быть
антецедентом местоимения,~ср.:

\begin{enumerate}
  \def\labelenumi{(\arabic{enumi})}
  \setcounter{enumi}{24}
  \item
        *Из Джона и Мери я больше люблю его.
\end{enumerate}

Согласно {[}Падучева~1985:~123{]}, данный запрет объясняется тем, что
антецедент местоимения должен в некотором фрагменте текста быть
смысловой темой, тогда как конъюнкт может составить тему лишь вместе с
другим конъюнктом. См.~также статью \underline{Коммуникативная структура
  предложения}.

\section{Отступления от симметричного
  канона}\label{ux43eux442ux441ux442ux443ux43fux43bux435ux43dux438ux44f-ux43eux442-ux441ux438ux43cux43cux435ux442ux440ux438ux447ux43dux43eux433ux43e-ux43aux430ux43dux43eux43dux430}

Идея симметрии лежит в основе сочинительного отношения
(см.~\underline{п.3.~Симметрия сочинения}). Однако наряду с проявлениями
этой симметрии на разных уровнях, при сочинении возможны и разнообразные
отклонения от симметричного образца. Ниже рассмотрены наиболее
существенные из таких отклонений и подведен итог тому, какие аспекты
симметрии принципиальны для характеризации конструкции как канонической
сочинительной.

\subsection{Необратимость при
  сочинении}\label{ux43dux435ux43eux431ux440ux430ux442ux438ux43cux43eux441ux442ux44c-ux43fux440ux438-ux441ux43eux447ux438ux43dux435ux43dux438ux438}

Традиционно считается, что порядок элементов в сочинительной конструкции
часто произволен и может меняться без существенных изменений смысла
(см.~\underline{п.3.3.4.~Обратимость конъюнктов}). В русской
грамматической традиции первым на обратимость при сочинении указал
А.~М.~Пешковский {[}Пешковский~1928/2001{]}; согласно Пешковскому,
обратимость --- принципиальная особенность, отличающая сочинение от
подчинения. Однако за последние десятилетия был обнаружен целый ряд
случаев, когда линейный порядок конъюнктов в сочинительной конструкции
фиксирован. Справедливым, по-видимому, будет даже утверждать, что
конъюнкты необратимы чаще, чем обратимы (хотя точная статистика здесь
вряд ли возможна).

Среди конструкций, в которых порядок компонентов фиксирован, принято
различать устойчивые сочинительные сочетания, отражающие типовые
представления о некотором фрагменте действительности (\textit{до и после},
\textit{любовь и ненависть}, \textit{причина и следствие}) и сочетания, не
являющиеся устойчивыми и связанные с конкретной ситуацией (\textit{Маша и
  ее брат}; \textit{стол и шкаф}) {[}Cooper,~Ross~1975{]},
{[}Лауфер~1987{]}, {[}Санников~2008{]}, {[}Урысон~2011{]}\footnote{Противопоставление
  устойчивых и ситуативно обусловленных сочетаний близко
  противопоставлению «естественной» и «случайной» конъюнкции (англ.
  \textit{natural} vs. \textit{accidental conjunction}), принятому в
  типологической литературе (см.~\underline{п.7.1.1}).}.

\underline{ПРИМЕЧАНИЕ}.~В работах, посвященных порядку конъюнктов в
сочинительной конструкции, рассматривается преимущественно сочинение с
участием союза \textit{и}. Это не случайно: помимо того, что \textit{и} --
самый частотный из сочинительных союзов (см.~статистику в статье
\underline{Союз}), его соединительная (в частности, перечислительная)
семантика удобна для использования линейного порядка иконически.
Иконический порядок, отражающий реальную хронологию событий, как раз и
является основным случаем фиксированного порядка при сочинении
{[}Якобсон~1983{]}~и~др. Союз \textit{и} важно рассмотреть в связи с
необратимостью еще и потому, что именно \textit{и} располагает к
обратимости конъюнктов (ср.~\textit{Маша и Петя} vs. \textit{Петя и Маша}),
поскольку при перечислительном употреблении связывает конъюнкты
семантически симметричным отношением (см.~также \underline{п.3.3}).
Помимо \textit{и}, смысловой симметрией и, соответственно, потенциальной
обратимостью обладают соединительные повторяющиеся союзы
\textit{и\ldots и}, \textit{ни\ldots ни}, разделительные союзы \textit{или},
\textit{или\ldots или}, \textit{то\ldots то} и некоторые другие (ср.
\textit{ни ты}, \textit{ни я} vs. \textit{ни я}, \textit{ни ты}; \textit{Маша или
  Петя} vs. \textit{Петя или Маша}). Приводимые ниже случаи необратимости
при сочинении применимы к данным союзам так же, как к союзу \textit{и}.
Напротив, многие другие сочинительные союзы, в особенности
противительные, выражают семантически асимметричное отношение и
обратимостью не обладают изначально. Так, союз \textit{но}
противопоставляет второй конъюнкт первому, но не наоборот, поэтому
\textit{красивая}, \textit{но глупая}~≠ \textit{глупая}, \textit{но красивая}.

\subsubsection{Порядок конъюнктов в устойчивых
  сочетаниях}\label{ux43fux43eux440ux44fux434ux43eux43a-ux43aux43eux43dux44aux44eux43dux43aux442ux43eux432-ux432-ux443ux441ux442ux43eux439ux447ux438ux432ux44bux445-ux441ux43eux447ux435ux442ux430ux43dux438ux44fux445}

Порядок конъюнктов в устойчивых сочетаниях регулируется следующими
основными «принципами» (по терминологии {[}Лауфер~1987{]}).

1.~Принцип предшествования

Реальное предшествование во времени, в пространстве или по порядку
предрасполагает к предшествованию соответствующего конъюнкта. Ср.
\textit{вчера}, \textit{сегодня}, \textit{завтра}; \textit{преступление и
  наказание}; \textit{начало и конец}; числа натурального ряда (\textit{1},
\textit{2}, \textit{3\ldots{}}), дни недели (\textit{понедельник},
\textit{вторник\ldots{}}), месяцы (\textit{январь}, \textit{февраль\ldots{}})
и~т.п.

2.~Принцип первичности

Первым выступает конъюнкт, выражающий объект или ситуацию, которые могут
считаться в каком-либо смысле первичными или исходными. Сюда относятся,
в частности:

2.1.~Причина~/~следствие: \textit{спрос и предложение}, \textit{вопрос и
  ответ} и~т.п.

2.2.~Проект~/~реализация: \textit{подготовка и осуществление}, \textit{слово
  и дело} и~т.п.

2.3.Семантически и формально исходное~/~производное: \textit{тигр и
  тигренок}, \textit{вчера и позавчера}, \textit{нужное и ненужное} и~т.п.

3.~Принцип рангового превосходства

Конъюнкт, выражающий объект более высокого ранга, предшествует
конъюнкту, выражающему объект более низкого ранга. Ср. \textit{мужчина и
  женщина}; \textit{братья и сестры} (гендерная иерархия); \textit{леди и
  джентльмены}; \textit{дамы и господа} (светская); \textit{отцы и дети};
\textit{стар и млад} (возрастная); \textit{хозяин и слуга}; \textit{президент
  и премьер-министр}; \textit{профессора}, \textit{преподаватели и студенты}
(административная); \textit{рабочий класс}, \textit{крестьянство и
  интеллигенция} (идеологическая) и~т.п.

4.~Принцип активности

Более активный участник ситуации предшествует более пассивному. Ср.
\textit{врач и пациент}, \textit{паровоз и вагоны} и~т.п.

5.~Принцип фиксации центра

Конъюнкт, обозначающий центральный объект, занимает первое место. Ср.
\textit{центр и периферия}, \textit{Солнце и планеты}, \textit{город и
  окраины}. В качестве центра могут выступать координаты речевого акта:
его субъект (Говорящий), место и время. Ср.: \textit{и}
\textless{}\textit{ни}\textgreater{} \textit{нашим}, \textit{и}
\textless{}\textit{ни}\textgreater{} \textit{вашим}; \textit{отечественные и
  зарубежные фильмы}; \textit{то тут}, \textit{то там}; \textit{теперь или
  никогда} и~т.п.

6.~Принцип преобладающего количества

Первое место в линейной последовательности занимает компонент, имеющий
признак «больше». Ср.: \textit{более или менее}; \textit{ни много}, \textit{ни
  мало}; \textit{долго ли}, \textit{коротко ли} и т.п.

7.~Принцип предпочтительной альтернативы

Конъюнкт, выражающий положительное явление, предшествует конъюнкту,
выражающему отрицательное явление. Ср.: \textit{за и против}; \textit{любовь
  и ненависть}; \textit{друг или враг} и т.п.

8.~Принцип антропоцентризма

Раньше располагается то, что важнее для человека с точки зрения
обеспечения условий его существования. Ср.: \textit{право и лево};
\textit{передний и задний}; \textit{нос и корма}; \textit{руки и ноги};
\textit{жизнь и смерть}; \textit{день и ночь} и т.п.

Перечисленные принципы объединены той идеей, что первая позиция в
сочинительной конструкции является с некоторой точки зрения наиболее
значимой и престижной. Исключение составляет принцип предшествования,
отражающий, скорее, идею упорядоченности, чем престижа.

Соблюдение данных принципов не является строго обязательным. Принципами
регулируется стандартный порядок в устойчивом сочинительном сочетании;
их нарушение ведет не к грамматической неправильности, а к
нестандартности. Так, для сочетания \textit{добро и зло} (выбранного
случайным образом) в Корпусе обнаружено 364 вхождения при стандартном
порядке и 39 вхождений при инвертированном порядке \textit{зло и добро}.

\subsubsection{Порядок конъюнктов в сочинительных сочетаниях, не
  являющихся
  устойчивыми}\label{ux43fux43eux440ux44fux434ux43eux43a-ux43aux43eux43dux44aux44eux43dux43aux442ux43eux432-ux432-ux441ux43eux447ux438ux43dux438ux442ux435ux43bux44cux43dux44bux445-ux441ux43eux447ux435ux442ux430ux43dux438ux44fux445-ux43dux435-ux44fux432ux43bux44fux44eux449ux438ux445ux441ux44f-ux443ux441ux442ux43eux439ux447ux438ux432ux44bux43cux438}

В сочинительных сочетаниях, не являющихся устойчивыми, фиксированность
порядка может быть связана, прежде всего, с двумя факторами.

1.~Объекты или ситуации следуют друг за другом во времени, в
пространстве или по порядку, а линейный порядок отражает эту
последовательность, т.е. является иконическим (в устойчивых
сочинительных сочетаниях иконический порядок также реализуется,
см.~\underline{п.4.1.1}, принципы предшествования и первичности). Ср.
хрестоматийный пример иконического порядка у Р.~Якобсона: \textit{veni},
\textit{vidi}, \textit{vici} {[}Якобсон~1983{]}, а также пример из
{[}Урысон~2011:~305{]}:

\begin{enumerate}
  \def\labelenumi{(\arabic{enumi})}
  \setcounter{enumi}{25}
  \item
        а.~Петербургский сезон начал замирать, и все понемногу разъезжались.
        --- б.~Все понемногу разъезжались, и петербургский сезон начал
        замирать.
\end{enumerate}

В (а) замирание сезона --- причина разъезда, в (б) разъезд --- причина
замирания сезона; эта разница в смысле выражена иконическим порядком
конъюнктов, отражающим тот факт, что причина в реальности предшествует
следствию.

2.~Порядок компонентов обусловлен анафорический отсылкой, содержащейся в
некотором конъюнкте, к какому-то из линейно предшествующих конъюнктов:
\textit{малина и другие ягоды}, \textit{Маша и ее}
\textless{}\textit{Машин}\textgreater{} \textit{брат}, ср. также:

\begin{enumerate}
  \def\labelenumi{(\arabic{enumi})}
  \setcounter{enumi}{26}
  \item
        Многие коллекционируют в детстве \textit{почтовые марки}, \textit{открытки
          и тому подобные вещи}. {[}«Наука и жизнь» (2006){]}
\end{enumerate}

Перестановка конъюнктов в этом случае ведет или к грамматической
неправильности, или к стилистической небрежности: *\textit{другие ягоды и
  малина}, *\textit{тому подобные вещи и марки},
\textsuperscript{?}\textit{Машин брат и Маша} (ср.~допустимое \textit{Машин
  брат и сама Маша}: за счет местоимения \textit{сам} второй конъюнкт
интерпретируется как содержащий отсылку к первому, а первый, наоборот,
как автономный и не содержащий отсылки).

\subsubsection{Взаимодействие факторов, отвечающих за порядок
  конъюнктов}\label{ux432ux437ux430ux438ux43cux43eux434ux435ux439ux441ux442ux432ux438ux435-ux444ux430ux43aux442ux43eux440ux43eux432-ux43eux442ux432ux435ux447ux430ux44eux449ux438ux445-ux437ux430-ux43fux43eux440ux44fux434ux43eux43a-ux43aux43eux43dux44aux44eux43dux43aux442ux43eux432}

Перечисленные факторы, ответственные за фиксированный порядок конъюнктов
(см.~\underline{п.4.1.1}, \underline{п.4.1.2}), разнообразно соотносятся
друг с другом.

Во-первых, порядок, продиктованный конкретной ситуацией, обычно важнее
порядка в устойчивых сочинительных сочетаниях. Так, в следующем
предложении порядок в пространстве «побеждает» административную
иерархию:

\begin{enumerate}
  \def\labelenumi{(\arabic{enumi})}
  \setcounter{enumi}{27}
  \item
        На фотографии слева направо: \textit{капитан Иванов}, \textit{майор
          Петров}, \textit{генерал Григорьев} (пример из {[}Лауфер~1987:~175{]})
\end{enumerate}

Во-вторых, в устойчивых сочинительных сочетаниях порядок может
определяться не одним, а сразу несколькими принципами. Так, в сочетании
\textit{король и королева} действуют одновременно иерархия по полу,
иерархия по общественному положению и принцип первичности.

В-третьих, в устойчивых сочетаниях разные принципы могут находиться в
противоречии друг с другом, и тогда выбор доминирующего принципа
осуществляется Говорящим. Ср.~\textit{прадед}, \textit{дед и отец} (принцип
предшествования) vs. \textit{отец}, \textit{дед и прадед} (принцип фиксации
центра). Ср.~также следующий пример, в котором принцип предшествования
«побежден» принципом фиксации центра:

\begin{enumerate}
  \def\labelenumi{(\arabic{enumi})}
  \setcounter{enumi}{28}
  \item
        Вывод один: компания переоценена в разы, якуты-менеджеры льют
        потихому-потихому, и \textit{сегодня} лили, и \textit{в пятницу}, и
        \textit{в четверг}, и даже \textit{в среду}. {[}коллективный. Золото
        Якутии :) (2010){]}
\end{enumerate}

Наконец, наряду с перечисленными факторами, необратимость порядка может
быть вызвана и некоторыми другими причинами. Это, в частности,
фонетический фактор, согласно которому более длинный конъюнкт обычно
следует за более коротким {[}Якобсон~1975{]}, {[}Урысон 2011{]}~и~др.
Ср.~стандартный пример (а) с не вполне стандартным (б):

\begin{enumerate}
  \def\labelenumi{(\arabic{enumi})}
  \setcounter{enumi}{29}
  \item
        а.~Оно будет удерживаться в прорези посредством \textit{упора} и
        \textit{язычков, разведённых в разные стороны}. {[}«Народное творчество»
        (2004){]} --- б.~Оно будет удерживаться в прорези посредством
        \textit{язычков, разведённых в разные стороны}, и \textit{упора}.
\end{enumerate}

\subsection{Сочинение составляющих разных
  категорий}\label{ux441ux43eux447ux438ux43dux435ux43dux438ux435-ux441ux43eux441ux442ux430ux432ux43bux44fux44eux449ux438ux445-ux440ux430ux437ux43dux44bux445-ux43aux430ux442ux435ux433ux43eux440ux438ux439}

В прототипическом случае сочиняются составляющие одинаковых категорий
(именная группа с именной группой, глагольная с глагольной и~т.д.,
см.~\underline{п.3.2}). Однако в ряду других аспектов симметрии
сочинения, требование категориального тождества соблюдается не строго.
Ср.~примеры сочинения составляющих разных категорий:

\begin{enumerate}
  \def\labelenumi{(\arabic{enumi})}
  \setcounter{enumi}{30}
  \item
        А сейчас я видела, как мой тринадцатилетний сын
        {[}\textit{спокойно}{]}ГНар и {[}\textit{безо всякого трепета}{]}ПредГ
        обращается с этой машиной. {[}«Даша» (2004){]}
  \item
        Вода в нём гладкая и дымная, ― такою она бывает {[}\textit{ранним
          утром}{]}ИГ или {[}\textit{после заката}{]}ПредГ. {[}К.~Г.~Паустовский.
        Орест Кипренский (1936){]}
\end{enumerate}

Достаточно строгим данное требование является только для подлежащего и
дополнений, ср.~*\textit{дом отошел} {[}\textit{брату}{]}ИГ \textit{и}
{[}\textit{к племяннику}{]}ПредГ, *\textit{мне нравится}
{[}\textit{плавание}{]}ИГ \textit{и} {[}\textit{играть в теннис}{]}ГГ, (хотя и
здесь возможны исключения, ср.: \textit{охотник} {[}\textit{петь}{]}ГГ
\textit{и} {[}\textit{до кур}{]}ПредГ (А.~Герцен)). Особое положение
подлежащего и дополнений объясняется, по-видимому, тем, что
применительно к ним тенденция к категориальному тождеству пересекается с
требованиями морфосинтаксической симметрии, согласно которым желательно
единообразное грамматическое оформление сочиняемых актантов
(см.~\underline{п.3.1}). Таким образом, категориальное тождество в
случае с актантами мало показательно и может быть объяснено не столько
собственно требованием категориального тождества при сочинении, сколько
указанными выше требованиями симметрии.

Как видно из примеров, отсутствие категориального тождества допустимо
при условии, что конъюнкты выступают в тождественной синтаксической
функции --- одного и того же члена предложения (ср.: \textit{спокойно и
  безо всякого трепета} --- обстоятельства образа действия, \textit{ранним
  утром или после заката} --- обстоятельства времени и~т.д.). Обратное
обычно тоже верно: если конъюнкты соответствуют разным членам
предложения, но составляющим одинаковых категорий, сочинение часто
затруднено: \textsuperscript{?}\textit{я курю только}
{[}\textit{вечерами}{]}ИГ \textit{и} {[}\textit{сигары}{]}ИГ;
\textsuperscript{?}\textit{Иван Иванович уехал} {[}\textit{в
  Германию}{]}ПредГ \textit{и} {[}\textit{на три месяца}{]}ПредГ
(об~исключениях см.~\underline{п.4.3.~Сочинение разных членов
  предложения}).

\subsection{Сочинение разных членов
  предложения}\label{ux441ux43eux447ux438ux43dux435ux43dux438ux435-ux440ux430ux437ux43dux44bux445-ux447ux43bux435ux43dux43eux432-ux43fux440ux435ux434ux43bux43eux436ux435ux43dux438ux44f}

Сочинение разных членов предложения допустимо в следующих случаях:
{[}Санников~1989:~15--25{]}, {[}Санников~2008:~111--119{]},
{[}Тестелец~2001:~212{]}.

1.~Сочиняются вопросительные слова, ср.~\textit{Кто и что говорил?}
Ср.~также:

\begin{enumerate}
  \def\labelenumi{(\arabic{enumi})}
  \setcounter{enumi}{32}
  \item
        На самом же деле абсолютно не важно: сколько готовились аресты и
        \textit{кто}, \textit{как и за что} попался. {[}«Новая газета» (2003){]}
\end{enumerate}

2.~Сочиняются кванторные слова, принадлежащие к одному и тому же разряду
(отрицательные, неопределенные и~т.д.), ср.:

\begin{enumerate}
  \def\labelenumi{(\arabic{enumi})}
  \setcounter{enumi}{33}
  \item
        В этой стране \textit{все и всем} недовольны (пример из
        {[}Санников~2008:~111{]})
  \item
        Запивая все это принесенной с собой водкой, мы сразу выяснили
        \textit{все и навсегда} ― от Гоголя до Ерофеева. {[}А.~Генис. Довлатов и
        окрестности (1998){]}
  \item
        Но таких явлений \textit{никто и никогда} не наблюдал. {[}В.~Н.~Комаров.
        Тайны пространства и времени (1995-2000){]}
  \item
        Их пленило в ней общинное владение, потому что \textit{кто-то и
          когда-то} сказал что-то в похвалу общинного владения
        \textless\ldots\textgreater. {[}Н.~А.~Добролюбов. Литературные мелочи
        прошлого года (1859){]}
\end{enumerate}

При сочинении неопределенных местоимений на \textit{кое}-, -\textit{либо},
-\textit{нибудь}, -\textit{то}, они должны оформляться одинаковым аффиксом,
ср. *\textit{чему-либо и как-нибудь}, *\textit{кое-кто и что-нибудь}.

Сюда же примыкает сочинение элементов, каждый из которых находится в
сфере действия отрицательной частицы \textit{не}:

\begin{enumerate}
  \def\labelenumi{(\arabic{enumi})}
  \setcounter{enumi}{37}
  \item
        Приехал \textit{не Иван и не в Москву} (пример из
        {[}Тестелец~2001:~212{]})
\end{enumerate}

3.~Сочиняются слова или словосочетания, вершины или зависимые которых
представляют собой однокоренные лексемы или, наоборот, контрастны в
лексико-семантическом отношении --- в зависимости от семантики союза:
соединительный \textit{и} требует сходства, противительный \textit{но} --
контрастности. Ср.:

\begin{enumerate}
  \def\labelenumi{(\arabic{enumi})}
  \setcounter{enumi}{38}
  \item
        а.~Я говорю \textit{с лингвистом и о лингвисте} \textless о лингвистике,
        *о художнике, *о языковеде\textgreater{}
\end{enumerate}

б.~Я говорю с лингвистом, но о художнике \textless*о лингвистике, *о
лингвисте, *о языковеде\textgreater{}

\begin{enumerate}
  \def\labelenumi{(\arabic{enumi})}
  \setcounter{enumi}{39}
  \item
        Я под всякими предлогами внушал ей одно: живи только \textit{для меня и
          мной} \textless\ldots\textgreater. {[}И.~А.~Бунин. Жизнь Арсеньева.
        Юность (1927-1933){]}
  \item
        Ведь эта система, как и ныне действующая, очевидно, опять же придумана
        \textit{врачами и для врачей}. {[}«Лебедь» (2003){]}
  \item
        \textless\ldots\textgreater{} это все было \textit{в одно время и на
          одной земле}, \textit{но как бы на разных улицах}. {[}А.~Лазарчук. Все,
        способные держать оружие... (1995){]}
\end{enumerate}

Сочинение разных членов предложения, проиллюстрированное в пунктах~1--3,
характеризуется тем, что отсутствие у конъюнктов синтаксической
однородности восполняется наличием однородности лексико-семантической
(по терминологии {[}Санников~1989{]}, {[}Санников~2008{]}, ср.~также
близкий подход в {[}Белошапкова~1977{]}, {[}Белошапкова~1989{]}).

В ряде случаев, однако, сочинение разных членов предложения возможно и в
отсутствие лексико-семантической однородности (см.~пункт 4).

4.~Сочиняются слова или словосочетания, которые не обладают ни
лексической, ни явной семантической однородностью, но которым
однородность навязывается самим фактом сочинения. Так, фраза \textit{Я
  курю вечерами и сигары}, вне контекста сомнительная, оказывается
допустима в качестве ответа на предложение выкурить сигарету утром. В
этом случае конъюнкты оказываются однородны прагматически --- постольку,
поскольку оба опровергают возможность принять прозвучавшее предложение.
Ср.~также следующий пример, где основанием для сочинения прямого
дополнения с косвенным и с обстоятельством служит тот факт, что каждое
характеризует характер «письма» как дружеский и неформальный:

\begin{enumerate}
  \def\labelenumi{(\arabic{enumi})}
  \setcounter{enumi}{42}
  \item
        Считаю себя вправе писать вам \textit{карандашом}, \textit{в постели и
          самое домашнее письмо} (А.~Блок)
\end{enumerate}

При сочинении такого рода, стилистически маркированном и требующем
специального контекста, выражается намерение Говорящего указать на
одноплановость разных аспектов ситуации с некоторой, уместной в данном
контексте, точки зрения {[}Пешковский~1928/2001{]},
{[}Апресян~и~др.~2010{]}\footnote{Условия~3 и 4, разрешающие сочинение
  разных членов предложения --- это два разных полюса одной "оси":
  бесспорное наличие лексико-семантической однородности у конъюнктов,
  подкрепленное тождеством их корней, с одной стороны (п.3), и
  семантико-прагматическая однородность, подсказанная только контекстом,
  с другой (п.4). Между этими полюсами имеется, однако, множество
  промежуточных случаев. Ср. \textit{говорю с композитором и о музыке
    \textless с писателем и о литературе} и~под.\textit{\textgreater{}}: у
  конъюнктов отсутствует лексическая однородность, но, независимо от
  контекста, наличествует семантическая.}.

\subsection{Вторичная союзная
  связь}\label{ux432ux442ux43eux440ux438ux447ux43dux430ux44f-ux441ux43eux44eux437ux43dux430ux44f-ux441ux432ux44fux437ux44c}

Под вторичной союзной связью понимается синтаксическая связь,
оформленная сочинительным союзом, в конструкциях следующего вида:

\begin{enumerate}
  \def\labelenumi{(\arabic{enumi})}
  \setcounter{enumi}{43}
  \item
        Теперь о возможности обобрать "нефтянку". \textit{Она есть}, \textit{но
          очень небольшая}. {[}«Еженедельный журнал» (2003){]}
  \item
        Термин "религия" сегодня \textit{используется очень широко во многих
          науках и публицистике}, \textit{но в очень узком смысле}
        \textless\ldots\textgreater. {[}«Адвокат» (2004){]}
  \item
        Хуже нет, как брюхо \textit{растравишь}, \textit{да попусту}.
        {[}А.~Солженицын. Один день Ивана Денисовича (1961){]}
  \item
        \textless\ldots\textgreater"Машина времени", наоборот, \textit{звучала},
        \textit{и вовсю}. {[}«Известия» (2002){]}
\end{enumerate}

Вторичной такая сочинительная связь называется потому, что она
«накладывается» на первичную подчинительную: союз здесь связывает
элементы, уже соединенные подчинительной связью. Ср.~результат опущения
сочинительного союза в примерах выше: \textit{она очень небольшая} (связка
\textit{есть}, в соответствии с общим правилам, меняется на нулевую),
\textit{используется в очень узком смысле}, \textit{растравишь попусту},
\textit{звучала вовсю}. Наиболее употребительны в конструкции с вторичной
связью противительные и соединительные союзы: \textit{но}, \textit{и},
\textit{да}.

В примерах ()--() конъюнкт, соответствующий зависимому в «первичной»
подчинительной конструкции, находится в постпозиции к вершине. Однако
это не обязательно: союз \textit{но} допускает при вторичной связи и
обратный порядок. Ср.:

\begin{enumerate}
  \def\labelenumi{(\arabic{enumi})}
  \setcounter{enumi}{47}
  \item
        \textit{Достаточно скромно}, \textit{но присутствует} российский капитал в
        нефтегазовом секторе. {[}«Известия» (2003){]}
\end{enumerate}

Соединительные союзы \textit{и} и \textit{да}, напротив, требуют прямого
порядка: *\textit{вовсю}, \textit{и звучала}; *\textit{попусту}, \textit{да
  растравишь}.

Синтаксическая функция «зависимого» конъюнкта может быть любой:
подлежащее (\textit{пришел}, \textit{но не он}), дополнение (\textit{принес},
\textit{но не книгу}), обстоятельство (\textit{ушел}, \textit{и без куртки}),
определение (\textit{принес книгу}, \textit{но не ту}), присказуемостное имя
(\textit{вернулся}, \textit{но пьяным}) {[}Прияткина~1990{]},
{[}Санников~2008:~120{]}.

Функциональное назначение вторичной сочинительной связи лежит, прежде
всего, в коммуникативной плоскости. Оно состоит в том, чтобы единое
утверждение разбить на две отдельные коммуникативные составляющие,
выполняющие одинаковую коммуникативную роль --- ремы
{[}Леонтьева~1981{]}, {[}Санников~2008{]}
(см.~\underline{Коммуникативная структура предложения}). Так, в
предложении \textit{"Машина времени" звучала}, \textit{и вовсю} наличествуют
две, синтаксически и просодически отделенные, ремы: \textit{звучала} и
\textit{вовсю}. Ввиду данного свойства конструкция с вторичной
сочинительной связью названа В.~З.~Санниковым коммуникативно однородной
-- в отличие от лексико-семантической однородности в случаях типа
\textit{все и всем недовольны}, \textit{кто и что говорил} и под.
(см.~\underline{п.4.3.~Сочинение разных членов предложения}).

Вместе с тем, конструкция с вторичной связью имеет с конструкцией типа
\textit{все и всем} \textit{недовольны} ряд сходств.

К сходствам относится тот факт, что в конструкции с вторичной связью
также, на первый взгляд, сочинены разные члены предложения,
т.е.~нарушена синтаксическая симметрия. Правда, это сходство до
некоторой степени поверхностное: при вторичной связи, как правило, можно
говорить об эллипсисе элемента, тождественного «первичной» вершине, в
составе «зависимого» конъюнкта; такой элемент обычно восстановим,
ср.~\textit{ушел}, \textit{и ушел без куртки}; \textit{принес книгу},
\textit{но} \textit{книгу не ту}, \textit{растравишь}, \textit{да растравишь
  попусту} и~т.п\footnote{Исключение составляют конструкции со связкой
  типа \textit{она есть, но небольшая}, ср.~*\textit{она есть, но есть
    небольшая.} Однако здесь невозможность восстановления связана с
  действием постороннего фактора --- общим правилом употребления связки в
  русском языке --- а не со спецификой собственно конструкции. Кроме
  того, затруднено восстановление полусвязочных глаголов типа
  \textit{появиться, существовать} и~под. Ср. \textit{вероятность этого
    существует, но} (\textit{\textsuperscript{?}существует})
  \textit{небольшая.} Однако и здесь действует посторонний фактор:
  восстановленное слово оказывается темой и получает кодирующий тему
  восходящий акцент (ср. \textit{ушел, но ушел}/ \textit{без
    куртки}\textbackslash), тогда как для полусвязочных глаголов позиция
  акцентированной темы не характерна. Ср.~сочетания типа
  \textit{существует мнение}/\textit{, что}; \textit{появилась надежда}/,
  \textit{что} и~т.п., где предпочтительным и наиболее естественным
  является акцент на существительном, а не на глаголе.}. При
эллиптической трактовке сочиненными в данной конструкции всегда
оказываются одинаковые члены предложения. Напротив, в случае типа
\textit{все и всем недовольны} аналогичное восстановление и,
соответственно, эллиптическая трактовка, ведущая к синтаксической
однофункциональности конъюнктов, часто затруднены, поскольку у
предполагаемого восстановленного предиката может оказаться незаполненной
обязательная синтаксическая валентность. Ср.:
\textsuperscript{?}\textit{этого никто не простит и этого никому не
  простит} (восстановлено из \textit{этого никто и никому не простит}). Тем
не менее, обе конструкции объединены тем, что на поверхностном уровне
отходят от синтаксического сочинительного канона, компенсируя это в
одном случае, семантической однородностью, в другом --- коммуникативной.

\subsection{Сочинение с элементами
  подчинения}\label{ux441ux43eux447ux438ux43dux435ux43dux438ux435-ux441-ux44dux43bux435ux43cux435ux43dux442ux430ux43cux438-ux43fux43eux434ux447ux438ux43dux435ux43dux438ux44f}

Имеется ряд разнородных конструкций, в которых, наряду с признаками
сочинения, присутствуют также признаки подчинения. Ниже рассматриваются
основные из таких конструкций.

\subsubsection{Изловчился и поймал (пса за
  хвост)}\label{ux438ux437ux43bux43eux432ux447ux438ux43bux441ux44f-ux438-ux43fux43eux439ux43cux430ux43b-ux43fux441ux430-ux437ux430-ux445ux432ux43eux441ux442}

Глагол \textit{изловчиться}, а также некоторые семантически близкие
глаголы (\textit{поднатужиться}, \textit{поднапрячься} и~др.) обладают
следующей особенностью: стандартным способом заполнения имеющейся у них
обязательной синтаксической валентности цели является сочинительное
сочетание с союзом \textit{и} {[}Богуславский~1996:~28--32{]}. Ср.:

\begin{enumerate}
  \def\labelenumi{(\arabic{enumi})}
  \setcounter{enumi}{48}
  \item
        И все же лиса \textit{изловчилась и дотянулась} \textit{до шеи орлана}.
        {[}В.~Бурлак. Хранители древних тайн (2001){]}
\end{enumerate}

Об обязательности соответствующей валентности свидетельствует
невозможность опустить второй конъюнкт, ср.~*\textit{И все же лиса
  изловчилась}.

\begin{enumerate}
  \def\labelenumi{(\arabic{enumi})}
  \setcounter{enumi}{49}
  \item
        Столько-де важных дел у государственного мужа, а все равно
        \textit{поднапрягся и вырвался из Москвы на денек}. {[}«Калининградская
        правда» (2003){]} --- *\textit{А все равно подняпрягся}
  \item
        Маленько \textit{поднатужилась и вылезла в отличницы}. {[}Н.~Подольский.
        Книга Легиона (2001){]} --- \textit{*Маленько поднатужилась}
\end{enumerate}

Такая сочинительная конструкция похожа на каноническое сочинение в том,
что сочиняются здесь синтаксически однофункциональные элементы --
сказуемые. Однако семантически отношение между конъюнктами напоминает
подчинение: один из конъюнктов заполняет валентность другого.

Согласно интерпретации И.~М.~Богуславского, указанное смешение
сочинительных и подчинительных свойств закрепилось в языке потому, что
при данных глаголах затруднен канонический способ заполнения
валентностей --- с помощью механизма управления. Так, при
\textit{изловчиться}, \textit{поднапрячься}, \textit{поднатужиться} часто
невозможно употребить инфинитив, стандартно заполняющий валентность
цели, например, у глаголов \textit{удаться}, \textit{смочь}, \textit{суметь}
(см.~\underline{Инфинитив}). Ср.: \textsuperscript{?}*\textit{лиса
  изловчилась дотянуться до шеи орлана}; *\textit{поднапрягся вырваться из
  Москвы на денек}; *\textit{поднатужилась вылезти в отличницы} vs.:

\begin{enumerate}
  \def\labelenumi{(\arabic{enumi})}
  \setcounter{enumi}{51}
  \item
        Благодаря такому разумному подходу его группе \textit{удалось завершить}
        свою работу раньше других. {[}«Computerworld» (2004){]}
  \item
        И он сам бы вам позвонил, но \textit{не смог придумать} никакого
        предлога\ldots{} и вышел. {[}Е.~Гришковец. ОдноврЕмЕнно (2004){]}
\end{enumerate}

Указанная интерпретация обсуждаемой конструкции не кажется
исчерпывающей, поскольку неясным остается, почему же при глаголах
\textit{изловчиться} и~т.п. затруднено заполнение валентности посредством
инфинитива. Тем более что запрет на инфинитив при этих глаголах не
абсолютен, ср.~примеры из Корпуса:

\begin{enumerate}
  \def\labelenumi{(\arabic{enumi})}
  \setcounter{enumi}{53}
  \item
        В бою с касогами кавказский богатырь \textit{изловчился хватить} его
        булавой по шлему. {[}Б.~Васильев. Ольга, королева русов (2002){]}
  \item
        Наконец, Маньяк \textit{изловчился прыгнуть}, как обычно, на соседний
        дом ― а дома-то уже и не было. {[}А.~Иванов. Географ глобус пропил
        (2002){]}
\end{enumerate}

Вопрос о том, как распределены между собой контексты с инфинитивом и с
сочинительной конструкцией и чем объясняется сама возможность заполнения
валентности сочинительными средствами, остается, таким образом, в
значительной степени открытым.

\subsubsection{Пошел и купил
  (книгу)}\label{ux43fux43eux448ux435ux43b-ux438-ux43aux443ux43fux438ux43b-ux43aux43dux438ux433ux443}

При сочинении с помощью союза \textit{и} двух финитных форм глагола иногда
допустим вынос элемента только из одного конъюнкта, нарушающий так
называемое «ограничение на сочиненную структуру» (см.
\underline{п.6.1.~Основные синтаксические свойства сочинительной
  конструкции}). Это сближает данную конструкцию с подчинением, при
котором подобного ограничения нет (см.~\underline{Сочинение и
  подчинение}). Ср.~(знаком "\_" обозначена исходная позиция выдвинутого
элемента):

\begin{enumerate}
  \def\labelenumi{(\arabic{enumi})}
  \setcounter{enumi}{55}
  \item
        а. \textless\ldots\textgreater взорви улицу по которой ходил \_ и
        думал о ней (Google)
\end{enumerate}

б.~Чая, который пошел бы и купил\_ , и еще раз повторно купил, мы на
дегустации сегодня не обнаружили. (Google)

в.~Одно из лучших воспоминаний детства. Утро, солнце в окно, рядом с
подушкой книга, которую читал \_ ... и заснул. (Google)

г.~Это чай, который ты ушел в кино и не допил \_. (пример из
{[}Зализняк,~Падучева~1979/2009{]})

Специфические признаки обсуждаемой конструкции --- т.е.~те условия, при
которых сочинение финитных глаголов начинает демонстрировать свойства
подчинения, --- до конца не изучены. Предположительно, возможность
нарушить «ограничение на сочиненную структуру» зависит от семантики
отношения, которым связаны финитные глаголы. Семантически асимметричное
отношение временного или причинного следования, как в примерах выше,
способствует нарушению, а семантически симметричное перечисление,
наоборот, препятствует нарушению, ср.:

\begin{enumerate}
  \def\labelenumi{(\arabic{enumi})}
  \setcounter{enumi}{56}
  \item
        *Это чай, который ты попробовал торт и не допил \_\_ .
\end{enumerate}

Это согласуется с тем, что сочинению прототипически свойственна
симметрия, а подчинению --- асимметрия (см.~\underline{п.3.~Симметрия
  сочинения}).

Сама по себе возможность нарушить «ограничение», при сохранении других
атрибутов, свойственных сочинению в русском языке (союз \textit{и},
синтаксическая однофункциональность конъюнктов), конечно, недостаточный
повод, чтобы трактовать такую конструкцию как подчинительную. Поэтому ее
принято относить к (неканоническим) сочинительным.

Важно, однако, отличать обсуждаемую конструкцию от формально близкой к
ней, но к сочинению уже не относящейся. Это конструкция с так
называемыми «двойными глаголами» типа \textit{возьму и скажу}, \textit{сижу
  скучаю} и~под. (см.~{[}Вайс~2000{]}, {[}Вайс~2003{]}), для которой
характерна частичная или полная десемантизация одного из глаголов. Так,
в сочетаниях \textit{возьму и скажу}, \textit{взял и заплакал} и~под. вся
конструкция обозначает одну ситуацию, а не две, тогда как при сочинении
каждый глагол обозначает собственную ситуацию, ср.~\textit{пошел бы и
  купил}, \textit{читал и заснул}, \textit{ходил и думал.} Впрочем, граница
между двойными глаголами и сочинением не жесткая, и для характеризации
пограничных случаев используется, наряду с критерием десемантизации, ряд
других тестов; в частности, в пользу сочинения при прочих равных
условиях свидетельствует наличие сочинительного союза. Так, с точки
зрения десемантизации сочетание \textit{ходил и думал}, выше названное
сочинительным, ближе к двойным глаголам, чем, например, явно
сочинительное \textit{читал и заснул}: \textit{ходил} в \textit{ходил и думал}
можно расценивать не как отдельную ситуацию, а как своего рода способ
осуществления действия, выраженного вторым глаголом, --- \textit{думать};
\textit{читать} и \textit{заснуть} между тем --- бесспорно разные ситуации.
Однако ввиду наличия союза \textit{ходил и думал} все же ближе к
сочинению, чем \textit{сижу скучаю}.

Русские «двойные глаголы» признаются близкими к так называемым
сериальным конструкциям, распространенным, в частности, в языках Африки
и Юго-Восточной Азии, а в европейских языках представленным лишь
единичными и небесспорными образцами. Об основных критериях
разграничения сочинения и сериализации (на материале двойных глаголов)
см.~указанные выше работы Д.~Вайса, а также
{[}Кибрик,~Подлесская~2009{]}.

\subsubsection{Еще одно слово, и я
  ухожу}\label{ux435ux449ux435-ux43eux434ux43dux43e-ux441ux43bux43eux432ux43e-ux438-ux44f-ux443ux445ux43eux436ux443}

В английском языке ряд подчинительных свойств обнаруживает конструкция с
сочинительным союзом \textit{and} и условно-следственным отношением между
клаузами:

\textit{You drink another can of beer}, \textit{and I am leaving} букв. ʽТы
пьешь еще одну кружку пива, и я ухожу'.

При переводе данного примера на русский язык (ср.~\textit{Еще одна кружка
  пива, и я ухожу}) условное прочтение также возможно, но более
естественным является понимание, при котором клаузы связаны отношением
временного следования: `после того как я выпью еще кружку, я уйду'.
Однако при другом лексическом наполнении данной конструкции и в русском
языке стандартной оказывается условная интерпретация. Ср.:

\begin{enumerate}
  \def\labelenumi{(\arabic{enumi})}
  \setcounter{enumi}{57}
  \item
        Еще одно слово, и я ухожу --- `если ты скажешь еще одно слово, я уйду'
  \item
        Еще шаг, и ты мертвец! ― крикнул он. {[}М.~Москвина. Небесные
        тихоходы: путешествие в Индию (2003){]}
\end{enumerate}

Как демонстрируется в {[}Culicover,~Jackendoff~1997{]}, в английском
языке данная конструкция сближается с подчинением по ряду своих
синтаксических особенностей. В частности, она допускает выдвижение
элементов отдельно из каждого конъюнкта, нарушающее «ограничение на
сочиненную структуру» (см.~\underline{Сочинение и подчинение}):

\begin{enumerate}
  \def\labelenumi{(\arabic{enumi})}
  \setcounter{enumi}{59}
  \item
        а.~This is the loot that you just identify \_ and we arrest the thief
        on the spot --- букв. `Это награбленное, которое едва вы опознаете \_ ,
        и мы тут же арестуем вора'
\end{enumerate}

б.~This is the thief that you just identify the loot and we arrest \_ on
the spot --- букв. `Это вор, которого едва вы опознаете награбленное, и
мы \_ тут же арестуем'

Примеры (а) и (б), по оценке авторов, не безупречны, но значительно
более приемлемы, чем результат аналогичного выдвижения из конструкции с
каноническим \textit{and}. Одновременно конструкция с условным \textit{and}
запрещает симметричное выдвижение элементов сразу из обоих конъюнктов,
что также является признаком подчинения (см. \underline{п.6.1.~Основные
  синтаксические свойства сочинительной конструкции} и статью
\underline{Сочинение и подчинение}):

\begin{enumerate}
  \def\labelenumi{(\arabic{enumi})}
  \setcounter{enumi}{60}
  \item
        \textsuperscript{??}This is the thief that you just point out \_ and
        we arrest \_ on the spot --- букв. `Это вор, на которого едва вы
        укажете \_ и мы тут же арестуем \_'
\end{enumerate}

Ср.~с результатом симметричного выдвижения при каноническом сочинении:
\textit{Торт}, \textit{который Вася купил\_}, \textit{а Маша съела\_},
\textit{был мой любимый}.

Кроме того, данная конструкция допускает меньшее разнообразие типов
эллипсиса, чем обычная сочинительная, при том что затрудненность
эллипсиса характерна для подчинения (см. \underline{п.6.2.~Сочинение и
  эллипсис}). В частности, в предложении с условным \textit{and} запрещен
так называемый гэппинг --- эллиптическое сокращение с образованием
внутреннего пробела (см.~\underline{Эллипсис}) --- допустимый при
каноническом сочинении. Ср.~допустимость гэппинга при каноническом
сочинении в (а) и его невозможность при условном прочтении and в (б):

\begin{enumerate}
  \def\labelenumi{(\arabic{enumi})}
  \setcounter{enumi}{61}
  \item
        а.~Big Louie stole another car radio and Little Louie stole the
        hubcaps --- букв. `Большой Луи украл еще один автомобильный
        радиоприемник, а Маленький Луи --- колесные колпаки'
\end{enumerate}

\begin{quote}
  б. *Big Louie steals one more car radio and Little Louie steals the
  hubcaps --- букв. `Большой Луи украдет еще один автомобильный
  радиоприемник, и Маленький Луи --- колесные колпаки'
\end{quote}

В русском языке, по сравнению с английским, демонстрация аналогичных
подчинительных свойств обсуждаемой конструкции затруднена:
ср.~невозможность литературного перевода на русский, с сохранением
синтаксиса конструкции, примеров (62а)--(62б). Это связано, по-видимому,
с тем, что в русском данная конструкция более идиоматична, поэтому ее
лексический состав и синтаксис строже «регламентированы». Тем не менее,
общее сходство русской конструкции с английской дает основания
предполагать, что союз \textit{и} в условном значении и в русском
отклоняется от канонического сочинения.

\subsubsection{Петя с Машей
  пришли}\label{ux43fux435ux442ux44f-ux441-ux43cux430ux448ux435ux439-ux43fux440ux438ux448ux43bux438}

Смешанные сочинительно-подчинительные свойства обнаруживает конструкция,
в которой позицию подлежащего занимают именные группы, соединенные
предлогом \textit{c}, а согласуемый с таким подлежащим глагол имеет форму
множественного числа. Ср.:

\begin{enumerate}
  \def\labelenumi{(\arabic{enumi})}
  \setcounter{enumi}{62}
  \item
        Но закон прорабатывали в правительстве, где \textit{Шойгу с Грызловым}
        обладают солидным весом. {[}«Завтра» (2003){]}
  \item
        Ко мне скоро \textit{жена с дочкой} приедут. {[}И.~Грекова. Перелом
        (1987){]}
\end{enumerate}

Данная конструкция отличается от трех предыдущих тем, что элементы
сочинения обнаруживаются в ней лишь в результате анализа, а по
формальным признакам конструкция является подчинительной (она содержит
не союз, а предлог, поэтому второй «конъюнкт» является зависимым по
отношению к первому, что не соответствует определению сочинения,
вынесенному в начало статьи). В конструкциях, описанных в
\underline{п.4.5.1}, \underline{п.4.5.2}, \underline{п.4.5.3}, ситуация
обратная: формальные признаки говорят о сочинении, и только анализ
обнаруживает подчинительные свойства.

Тем не менее, сходство с сочинением у конструкции \textit{Петя с Машей
  пришли} столь велико, что она часто рассматривается под рубрикой
«сочинение», ср., в частности, «комитативное сочинение» в {[}Haspelmath
2007:~29--33{]}. Основные сочинительные признаки у данной конструкции
следующие {[}Архипов~2009:~358--50,~159--164{]}, {[}Haspelmath~2007{]},
{[}Подлесская~2012{]} и~др.:

1.~Морфологический признак: форма множественного числа сказуемого, как
если бы подлежащее представляло собой сочиненную группу.

2.~Синтаксические признаки:

2.1.~Невозможно задать вопрос только к одному элементу конструкции,
ср.~*\textit{Петя с кем пришли?} *\textit{С кем Петя пришли?} При
каноническом сочинении аналогично: *\textit{Петя и кто пришли?} При
каноническом подчинении с помощью предлога \textit{с} данный запрет
отсутствует: \textit{С кем Петя пришел?}

2.2.~Невозможно дистантное расположение компонентов, ср.~*\textit{Петя
  пришли с Машей}. При каноническом сочинении аналогично: *\textit{Петя
  пришли и Маша}. При подчинении запрет отсутствует: \textit{Петя пришел с
  Машей}.

3.~Коммуникативно-просодический признак: затруднено выделение компонента
конструкции в отдельную --- контрастивную --- коммуникативную
составляющую. Ср. сомнительность контрастивного акцента на втором
компоненте конструкции: \textsuperscript{?}\textit{Петя с Машей пришли}
(\textit{а не с Мишей}). Аналогично при сочинении:
\textsuperscript{?}\textit{Петя и Маша пришли} (\textit{а не Петя и Миша}).
При каноническом подчинении запрет отсутствует: \textit{Петя пришел с
  Машей} (\textit{а не с Мишей}).

Особая трактовка обсуждаемой конструкции предлагается в статье
{[}Yuasa,~Sadock~2002{]}: конструкция названа подчинительной на
синтаксическом уровне и сочинительной --- на семантическом уровне (см.~об
этом также \underline{Сочинение и подчинение}). Такая трактовка, с одной
стороны, позволяет решить упомянутую выше трудность: поскольку именные
группы связаны не союзом, а предлогом, конструкция не отвечает
традиционному (синтаксическому) определению сочинения. Вместе с тем,
разграничение синтаксического и семантического сочинения обязывает
сформулировать отчетливое определение семантического сочинения
(противопоставленного синтаксическому!) --- но его ни в грамматической
традиции, ни в указанной статье не находим. Кроме того, такое
разграничение обязывало бы решить, какие из формальных признаков
конструкции свидетельствуют о ее синтаксическом статусе, а какие --- о
семантическом. Так, невозможность вопросительного выноса, объединяющая
данную конструкцию с канонической сочинительной, выше включена в число
синтаксических признаков. Однако характеризация конструкции как
семантически сочинительной потребовала бы, по-видимому, интерпретации
данного признака как семантического. Таким образом, сегодняшнее
состояние вопроса заставляет ограничиться более традиционной трактовкой,
согласно которой обсуждаемая конструкция признается неканонической
сочинительной. См.~о ней также в статье \underline{Именная группа}.

Синтаксическое сходство комитативного маркера с сочинительным
соединительным союзом типично для языков мира {[}Haspelmath~2007:~29{]}
и~др. Это связано с тем, что собственно комитативная конструкция
(\textit{Петя пришел с Машей}) и каноническое сочинение (\textit{Петя и Маша
  пришли}) близки семантически.

Итак, рассмотренные в~п.4.5.1--п.4.5.4 конструкции отличаются от
канонического сочинения тем, что сочинительные свойства в них совмещены
с подчинительными.

\subsection{Каноническое vs. неканоническое сочинение
  (итоги)}\label{ux43aux430ux43dux43eux43dux438ux447ux435ux441ux43aux43eux435-vs.-ux43dux435ux43aux430ux43dux43eux43dux438ux447ux435ux441ux43aux43eux435-ux441ux43eux447ux438ux43dux435ux43dux438ux435-ux438ux442ux43eux433ux438}

Поскольку словосочетания и предложения, оформляемые сочинительными
союзами, столь разнообразно отклоняются от симметричного образца, должен
быть решен вопрос о том, какие из этих отклонений релевантны для
отнесения конструкции к каноническому или неканоническому сочинению. В
рамках принятого выше, традиционного синтаксического подхода к сочинению
(см.~\underline{п.1.~Подходы к определению}) ответ на этот вопрос
следующий.

Необратимость при сочинении (см.~\underline{п.4.1}) --- явление настолько
частотное, что обратимость не может считаться конституирующим свойством
сочинительной конструкции. Тем самым, сочинительные сочетания, порядок
слов в которых фиксирован, нет основания выносить за рамки канонического
сочинения.

Аналогично, каноническим признается сочинение составляющих разных
категорий (\underline{п.4.2}); такое сочинение возможно при условии, что
конъюнкты соответствуют одному и тому же члену предложения,
т.е.~синтаксически однофункциональны. Синтаксическая
однофункциональность, таким образом, --- наиболее значимый аспект
симметрии сочинения.

Соответственно, отсутствие синтаксической однофункциональности в
конструкции с сочинительным союзом дает основания считать такую
конструкцию неканонической сочинительной. В пользу того, что это
все-таки сочинение, говорит, во-первых, отсутствие подчинительной связи
между конъюнктами, во-вторых, наличие сочинительного союза и, в-третьих,
наличие семантической или коммуникативной однородности, как бы
компенсирующей отсутствие однородности синтаксической
(см.~\underline{п.4.3.~Сочинение разных членов предложения} и
\underline{п.4.4.~Вторичная союзная связь}). В пользу того, что это
сочинение неканоническое, говорит тот факт, что в наиболее частотном и
прототипическом случае сочиняются одинаковые члены предложения, поэтому
такой случай естественно принять за канон.

Очевидно неканоническими являются, кроме того, особые конструкции,
совмещающие сочинительные свойства с подчинительными
(см.~\underline{п.4.5.~Сочинение с элементами подчинения}). Одну из
таких конструкций --- сочинение финитных глаголов, при котором может
нарушаться «ограничение на сочиненную структуру», --- следует отличать от
конструкции с «двойными глаголами» (см.~\underline{п.4.5.2}), к
сочинению уже не относящейся.

\section{Союзное~vs. бессоюзное
  сочинение}\label{ux441ux43eux44eux437ux43dux43eux435-vs.-ux431ux435ux441ux441ux43eux44eux437ux43dux43eux435-ux441ux43eux447ux438ux43dux435ux43dux438ux435}

Сочинительная связь может оформляться союзом (см.~\underline{Союз},
\underline{Сочинительные союзы}), а может не иметь формального
показателя и выражаться только соположением конъюнктов и интонацией. В
последнем случае имеет место бессоюзное, или асидентическое, сочинение.
Ср.:

\begin{enumerate}
  \def\labelenumi{(\arabic{enumi})}
  \setcounter{enumi}{64}
  \item
        С детьми регулярно занимались \textit{дефектологи}, \textit{логопеды},
        \textit{психолог}. {[}«Вопросы психологии» (2004){]}
\end{enumerate}

О бессоюзном сочинении традиционно говорят в связи с простым
предложением. Вопрос о существовании данного явления в сфере сложного
предложения является спорным. Согласно точке зрения, принятой в
русистике, бессоюзная связь клауз представляет собой особый вид
соединения, отличный от сочинения и подчинения {[}Белошапкова~1967{]},
{[}Грамматика~1980{]}, {[}Ширяев~1986{]} и~др. Согласно другой позиции,
бессоюзные сложные предложения могут быть сочинительными или
подчинительными, в зависимости от своих синтаксических свойств
{[}Тестелец~2001{]}, {[}Haspelmath~2007~и~др.{]}. Последняя точка зрения
обосновывается тем, что клаузы, будучи составляющими, могут или
полностью входить одна в другую (при подчинении), или вовсе не
пересекаться (при сочинении), и никакой третий способ их соединения
логически невозможен {[}Тестелец~2001:~264{]}. Другой аргумент состоит в
том, что отнесение бессоюзия в простом предложении и в сложном к
явлениям разной природы противоречит интуиции {[}Санников~2007:~339{]}.
О соотношении сочинения, подчинения и бессоюзия см.~также
\underline{Сочинение и подчинение}.

В простом предложении бессоюзное соединение синтаксически
однофункциональных элементов тоже не всегда может быть признано
сочинением. К сочинению обычно не относят конструкции, в которых,
несмотря на видимое синтаксическое подобие, налицо семантическая
иерархизованность элементов, характерная для подчинения
{[}Кодзасов,~Саввина~1987:~153{]}. Это конструкции с неоднородными
определениями (\textit{высокая каменная стена}, \textit{старинный рижский
  фарфор}), с семантикой уточнения или пояснения (\textit{приходи в среду в
  пять}; \textit{дай другой}, \textit{более светлый}, \textit{галстук}).

Бессоюзная сочинительная связь выражает, как правило, значение
перечисления и в простом предложении семантически близка союзу \textit{и}
(см.~\underline{Сочинительные союзы~/ п.3.1.1.~\textit{И}~перечисления})
Основное семантическое отличие бессоюзного перечисления от перечисления
с участием \textit{и} состоит в том, что \textit{и} всегда выражает
законченность перечисления {[}Санников~2008{]}, {[}Урысон~2011{]}. Ср.:

\begin{enumerate}
  \def\labelenumi{(\arabic{enumi})}
  \setcounter{enumi}{65}
  \item
        Я хочу подчеркнуть, что некоторые страны, такие, скажем, как Норвегия
        или Австралия, специализируются на производстве тех же первичных
        продуктов. Норвегия― это \textit{рыба}, \textit{нефть и газ}. Австралия ―
        \textit{шерсть}, \textit{мясо}. {[}«Еженедельный журнал» (2003){]}
\end{enumerate}

В данной последовательности предложений высказывание про Норвегию
содержит союз \textit{и}, а в высказывании про Австралию использовано
бессоюзное перечисление. Соответственно, только в первом случае
перечисление воспринимается как законченное: `рыбой, нефтью и газом
исчерпываются те первичные продукты, на производстве которых
специализируется Норвегия'.

\section{Синтаксис
  сочинения}\label{ux441ux438ux43dux442ux430ux43aux441ux438ux441-ux441ux43eux447ux438ux43dux435ux43dux438ux44f}

\subsection{Основные синтаксические свойства сочинительной
  конструкции}\label{ux43eux441ux43dux43eux432ux43dux44bux435-ux441ux438ux43dux442ux430ux43aux441ux438ux447ux435ux441ux43aux438ux435-ux441ux432ux43eux439ux441ux442ux432ux430-ux441ux43eux447ux438ux43dux438ux442ux435ux43bux44cux43dux43eux439-ux43aux43eux43dux441ux442ux440ux443ux43aux446ux438ux438}

Прототипическая сочинительная конструкция обладает рядом синтаксических
свойств, отличающих ее от подчинительной конструкции. В их числе
(подробнее см.~статью \underline{Сочинение и подчинение}).

1.~Сочинительный союз не может занимать в сочиненной группе
препозитивную позицию {[}Greenbaum~1969:~29{]}, {[}Van~Oirsouw~1987{]},
{[}Тестелец~2001{]}~и~др., ср.~\textit{дом и сад} vs. *\textit{и сад дом},
\textit{Он говорил очень убедительно}, \textit{но}
\textless{}\textit{а}\textgreater{} \textit{у меня возникли сомнения} vs.
*\textit{Но} \textless{}\textit{а}\textgreater{} \textit{у меня возникли
  сомнения}, \textit{он говорил очень убедительно} (возможно только при
отнесенности \textit{но} к предтексту). Двойные и повторяющиеся союзы
линейно примыкают к каждому конъюнкту (в русском языке --- предшествуют
ему, ср.~\textit{и дом}, \textit{и сад}; \textit{не только сад}, \textit{но и
  дом}), однако такого сочинительного союза, который состоял бы из одного
компонента и располагался бы в препозиции к сочиненной группе, в языках
мира не засвидетельствовано {[}Haspelmath~2007{]}.

2.~Ни одна из сочиненных клауз не может быть линейно вложена внутрь
другой. Ср.~пример из {[}Пешковский~1928/2001{]}: *\textit{Он не пошел},
\textit{и у него болит голова}, \textit{в школу}.

3.~При сочинении обычно невозможно применение к сочиненной группе таких
операций, которые включают в сферу действия только один конъюнкт. В
частности, нельзя задать вопрос к одному конъюнкту, оставив вне сферы
действия вопроса другой. Ср.: *\textit{Кто и Маша пришли?} *\textit{Кто
  рыжий}, \textit{а Вася блондин?} (вопрос к предложению \textit{Петя рыжий},
\textit{а Вася блондин}).

Допустимо, однако, симметричное применение операций, затрагивающее оба
конъюнкта. Так, к сложносочиненному предложению \textit{Маша купила
  конфеты}, \textit{а Петя их съел} можно задать вопрос \textit{Что Маша
  купила \_} , \textit{а Петя съел \_ ?}

Данное свойство сочинения впервые было обнаружено Дж.~Р.~Россом
{[}Ross~1967/1986{]} и получило название «ограничение на сочиненную
структуру» (англ. Coordinate Structure Constraint).

4.~При вложении сложносочиненного предложения в более крупную клаузу,
требующую от подчиненного предложения некоторого формального изменения
(например, постановки сказуемого в определенном наклонении), данное
изменение происходит с обоими конъюнктами. Ср.~\textit{Ты вернешься},
\textit{а я уйду} vs. \textit{Я хочу}, \textit{чтобы ты вернулся}
(\textit{*вернешься}), \textit{а я ушел} (\textit{*уйду}). Для сравнения, при
подчинении соответствующее изменение затрагивает только главную клаузу:
\textit{Ты вернешься}, \textit{когда я уйду} vs. \textit{Я хочу}, \textit{чтобы
  ты вернулся}, \textit{когда я уйду} (\textit{*ушел})
{[}Зализняк,~Падучева~1975{]}.

\subsection{Сочинение и
  эллипсис}\label{ux441ux43eux447ux438ux43dux435ux43dux438ux435-ux438-ux44dux43bux43bux438ux43fux441ux438ux441}

Сочинение связано с \underline{эллипсисом} по ряду оснований.

Во-первых, само описание механизма сочинения в некоторых случаях
затруднительно без привлечения аппарата эллипсиса. Речь идет, в первую
очередь, о случаях, когда в сочинении участвуют синтаксические единицы,
не являющиеся (на поверхностном уровне) каноническими составляющими.
Ср.~(квадратными скобками в примерах ниже отмечены границы конъюнктов):

\begin{enumerate}
  \def\labelenumi{(\arabic{enumi})}
  \setcounter{enumi}{66}
  \item
        Но уже не {[}Иван прислуживал королю{]}, а {[}король солдату{]}.
        {[}Л.~А.~Чарская. Дуль-Дуль, король без сердца (1912){]}
  \item
        Их родители не смогли бы тогда {[}сожалеть о сходстве{]} или
        {[}радоваться различию{]} поколений. (пример из
        {[}Кодзасов,~Саввина~1987:~150{]})
  \item
        Государь ездил {[}в сентябре на Дон{]}, {[}в октябре ― в Одессу{]} и
        {[}в ноябре ― в Польшу{]}. {[}А.~Ф.~Редигер. История моей жизни
        (1918){]}
\end{enumerate}

В (67) полная клауза сочинена с неполной клаузой, в которой отсутствует
глагол. В (68) сочинены фрагменты глагольных групп, не включающие общее
определение. В (69) сочинены пары обстоятельств, соподчиненные одному
глаголы. Во всех примерах (67)--(69) по крайней мере один из конъюнктов
невозможно трактовать как каноническую составляющую.

Существование таких «неконституентных» (non-constituent) конъюнктов
представляет определенную трудность для общей теории грамматики,
поскольку, согласно распространенной точке зрения, составляющая является
базовой синтаксической единицей, которой оперируют синтаксические
механизмы. Однако сочинение не-составляющих в большинстве случаев может
быть сведено к сочинению составляющих при помощи теоретического аппарата
эллипсиса. Так, в предложениях (67)--(69) конъюнкты окажутся
каноническими составляющими, если принять, что (67)--(69) получены из
глубинных структур (70)--(72) посредством эллиптического сокращения
одной из тождественных словоформ или цепочек словоформ (выделена
курсивом):

\begin{enumerate}
  \def\labelenumi{(\arabic{enumi})}
  \setcounter{enumi}{69}
  \item
        Но уже не {[}Иван прислуживал королю{]}, а {[}король
        \textit{прислуживал} солдату{]}.
  \item
        Их родители не смогли бы тогда {[}сожалеть о сходстве
        \textit{поколений}{]} или {[}радоваться различию поколений{]}.
  \item
        {[}Государь ездил в сентябре на Дон{]}, {[}\textit{государь ездил} в
        октябре в Одессу{]} и {[}\textit{государь ездил} в ноябре в Польшу{]}.
\end{enumerate}

Во-вторых, связь сочинения с эллипсисом прослеживается в том, что
некоторые виды эллипсиса, допустимые при сочинении, затруднены при
подчинении. Сюда относится так называемый гэппинг (англ. gapping), или
сокращение с образованием внутреннего пробела. Ср.:

\begin{enumerate}
  \def\labelenumi{(\arabic{enumi})}
  \setcounter{enumi}{72}
  \item
        Миша играл на рояле, \textit{а Маша --- на скрипке}. --- сочинение
\end{enumerate}

vs.

\begin{enumerate}
  \def\labelenumi{(\arabic{enumi})}
  \setcounter{enumi}{73}
  \item
        Миша играл на рояле, \textsuperscript{?}когда~/
        \textsuperscript{*}хотя /~\textsuperscript{*}несмотря на то что
        \ldots{} \textit{Маша --- на скрипке}. --подчинение
\end{enumerate}

Аналогично для «подъема правого узла» (англ. Right Node Raising) --
сокращения крайней правой составляющей в первом конъюнкте:

\begin{enumerate}
  \def\labelenumi{(\arabic{enumi})}
  \setcounter{enumi}{74}
  \item
        Маша приготовила, а Ваня съел рис. --- *Ваня съел после того, как Маша
        приготовила рис.
  \item
        Общественные объединения теряют, а \textless*хотя,
        *когда\ldots\textgreater{} партии приобретают право выдвижения
        кандидатов на выборах \textless\ldots\textgreater. {[}«Российская
        газета» (2003){]}
\end{enumerate}

Вместе с тем, открытым остается вопрос о том, каков масштаб участия
эллипсиса в сочинении. Крайняя точка зрения усматривает эллипсис в любом
несентенциальном сочинении, сводя, тем самым, почти (см.~ниже) любое
сочинение к сентенциальному. Согласно такой позиции, предложения (77а) и
(78а) получены из предложений (77б) и (78б) посредством эллиптического
преобразования, известного как «сочинительное сокращение» (англ.
conjunction reduction):

\begin{enumerate}
  \def\labelenumi{(\arabic{enumi})}
  \setcounter{enumi}{76}
  \item
        а.~Иван и Мария приехали издалека. --- б.~Иван приехал издалека и Мария
        приехала издалека.
  \item
        а.~Я позвоню тебе сегодня или завтра. --- б.~Я позвоню тебе сегодня или
        я позвоню тебе завтра.
\end{enumerate}

Сегодня, однако, такой крайней трактовки придерживаются лишь немногие
исследователи (см., например, {[}Wilder~1994{]}, {[}Wilder~1996{]},
{[}Wilder~1997{]}). Основные аргументы против сочинительного сокращения
состоят в следующем {[}Падучева~1974{]}, {[}Кодзасов,~Саввина~1989{]},
{[}Haspelmath~2007{]}, {[}Казенин~2011{]} и~др.:

1.~Эллиптичными невозможно считать конструкции, требующие множественных
актантов, ср.:

\begin{enumerate}
  \def\labelenumi{(\arabic{enumi})}
  \setcounter{enumi}{78}
  \item
        Маша и Катя похожи (*Маша похожа и Катя похожа)
  \item
        Старший и младший брат владеют \textit{разными языками} ( ≠ `Старший
        брат владеет разными языками и младший брат владеет разными языками')
\end{enumerate}

2.~Против эллипсиса подлежащего в конструкциях, где союз связывает
глагольные группы, свидетельствует поведение кванторных слов в позиции
подлежащего (\textit{Кто-то зашел в комнату и включил свет}). В этом
случае повтор квантора во втором конъюнкте (возникающий при
восстановлении предполагаемого эллипсиса) приводит к интерпретации,
отличной от смысла того предложения, где квантор выражен только один
раз. А именно, при повторе квантора предложение допускает две
интерпретации: ту, при которой субъекты при сочиненных предикатах
совпадают, и ту, при которой они различны; в отсутствие же повтора
единственно возможным является совпадение субъектов
{[}Казенин~2011{]}~и~др. Ср.:

\begin{enumerate}
  \def\labelenumi{(\arabic{enumi})}
  \setcounter{enumi}{80}
  \item
        а.~Кто завтра отдыхает и кто стартует? --- б.~\textsuperscript{\#}Кто
        завтра {[}отдыхает{]} и {[}стартует{]}?
  \item
        а.~Пять человек сняли свои кандидатуры и пять человек приняли участие
        в выборах --б.~\textsuperscript{\#}Пять человек сняли свои кандидатуры
        и приняли участие в выборах
\end{enumerate}

Предложение (81б), в отличие от (81а), аномально, поскольку единственно
возможная интерпретация --- совпадение субъектов --- семантически
исключена.

За пределами названных и некоторых других случаев, трактовка
сочинительной конструкции через трансформацию сочинительного сокращения
априори возможна; однако ее оправданность и теоретическая ценность
остаются предметом дискуссии.

О связи сочинения с эллипсисом см.~также в статье \underline{Эллипсис}.

\subsection{Согласование с сочиненной
  группой}\label{ux441ux43eux433ux43bux430ux441ux43eux432ux430ux43dux438ux435-ux441-ux441ux43eux447ux438ux43dux435ux43dux43dux43eux439-ux433ux440ux443ux43fux43fux43eux439}

При наличии в предложении сочиненных членов возможны колебания в форме
слов, связанных с конъюнктами согласовательной связью
(см.~\underline{Согласование}). Такие колебания могут иметь место в
следующих трех случаях:

1.~согласование сказуемого с сочиненным подлежащим в числе (\textit{во
  всем виден} \textless{}\textit{видны}\textgreater{} \textit{расчет и
  целеустремленность}); особыми правилами регулируется, кроме того,
согласование сказуемого с сочиненным подлежащим в лице (\textit{остаемся}
\textless{}\textsuperscript{?}\textit{остаешься}\textgreater{} \textit{ты и
  я} vs. \textit{остаешься не только ты, но и я});

2.~согласование атрибутивного прилагательного с сочиненными
существительными в числе (\textit{роскошный}
\textless{}\textit{роскошные}\textgreater{} \textit{костюм и рубашка});

3.~согласование в числе в группе с сочиненными прилагательными
(\textit{писатели старшего и среднего поколения}
\textless{}\textit{поколений}\textgreater).

Во всех трех случаях возможно также согласование по категории рода.
Данный вид согласования в настоящей статье отдельно не рассматривается,
поскольку согласование в роде имеет место только при единственном числе
мишени согласования (сказуемого или прилагательного), и в этом последнем
случае правило согласования в роде совпадает с основным правилом
согласования в числе. А именно, сказуемое или прилагательное может
согласовываться только с линейно ближайшим конъюнктом,
ср.~\textit{преподавалась} \textless{}\textit{*-лось}\textgreater{}
\textit{экономика и право}, \textit{красивая}
\textless{}\textit{*-ый}\textgreater{} \textit{юбка и жакет} (подробнее
см.~\underline{п.6.3.1.1}).

Ниже рассмотрены правила согласования для всех трех перечисленных
случаев~1,~2 и~3 (в числе и лице для случая~1, в числе для случаев~2
и~3).

\subsubsection{Согласование сказуемого с сочиненным
  подлежащим}\label{ux441ux43eux433ux43bux430ux441ux43eux432ux430ux43dux438ux435-ux441ux43aux430ux437ux443ux435ux43cux43eux433ux43e-ux441-ux441ux43eux447ux438ux43dux435ux43dux43dux44bux43c-ux43fux43eux434ux43bux435ux436ux430ux449ux438ux43c}

\paragraph{Согласование в
  числе}\label{ux441ux43eux433ux43bux430ux441ux43eux432ux430ux43dux438ux435-ux432-ux447ux438ux441ux43bux435}

Сказуемое при сочиненном подлежащем может иметь форму как множественного
числа:

\begin{enumerate}
  \def\labelenumi{(\arabic{enumi})}
  \setcounter{enumi}{82}
  \item
        Международная торговля и туризм \textit{изменили} положение дел
        \textless\ldots\textgreater. {[}Рецепты национальных кухонь:
        Скандинавская кухня (2000-2005){]}
\end{enumerate}

-- так и единственного; в последнем случае контролером согласования
всегда является линейно ближайший к сказуемому конъюнкт:

\begin{enumerate}
  \def\labelenumi{(\arabic{enumi})}
  \setcounter{enumi}{83}
  \item
        Последнее письмо ко мне Лидия Корнеевна написала уже в 1969 году,
        когда у неё \textit{возникло} \textless{}\textit{*-а}\textgreater{}
        желание и необходимость подготовить 3-е издание книги "В лаборатории
        редактора". {[}А.~Мильчин. В лаборатории редактора Лидии Чуковской
        (2001){]}
\end{enumerate}

Поэтому колебания в числе сказуемого возможны только там, где ближайший
к сказуемому конъюнкт имеет форму единственного числа. При множественном
числе ближайшего конъюнкта не возникает ситуации выбора: и ближайшее
подлежащее, и вся сочиненная группа одинаково требуют постановки
сказуемого в форму множественного числа:

\begin{enumerate}
  \def\labelenumi{(\arabic{enumi})}
  \setcounter{enumi}{84}
  \item
        С принятием христианства языческая символика и \textit{обряды}
        \textit{были запрещены}, но не искоренены полностью. {[}«Первое
        сентября» (2003){]}
\end{enumerate}

Выбор модели согласования (согласование с ближайшим конъюнктом и, в
случае если он имеет форму единственного числа, единственное число
сказуемого vs. согласование со всей сочиненной группой и множественное
число сказуемого) зависит от ряда семантических и синтаксических
факторов.

Основной семантический фактор состоит в том, что согласование со всей
сочиненной группой при прочих равных вероятнее в том случае, когда
сочиненное подлежащее имеет более одного референта, что, в свою очередь,
зависит от семантики союза. Ср.~\textit{Приходили}
\textless{}\textit{*-л}\textgreater{} \textit{Петя и Коля} vs.
\textit{Приходил} \textless{}\textsuperscript{?}\textit{-и}\textgreater{}
\textit{то ли Петя}, \textit{то ли Коля}.

Данная семантическая закономерность может подкрепляться или, наоборот,
нарушаться под действием ряда других семантико-синтаксических факторов,
среди которых можно выделить сильные и слабые
{[}Санников~2008:~151--160{]}, см.~также {[}Пешковский~1928/2001{]},
{[}Гвоздев~1973{]}, {[}Перетрухин~1979{]}, {[}Грамматика~1980{]},
{[}Граудина~и~др.~2008{]}, {[}Corbett~1979{]} и~др. К сильным факторам,
обычно однозначно определяющим модель согласования, относятся следующие.

1.~Кореферентность подлежащих предписывает согласование с ближайшим
подлежащим. Ср.:

\begin{enumerate}
  \def\labelenumi{(\arabic{enumi})}
  \setcounter{enumi}{85}
  \item
        В понедельник в Москву \textit{прибыл}
        \textless*\textit{прибыли}\textgreater{} вице-премьер и министр
        экономики Болгарии Николай Василев. {[}«Известия» (2002){]}
\end{enumerate}

Данный фактор, очевидно, является сильным потому, что он непосредственно
отражает указанную выше зависимость модели согласования от количества
референтов подлежащего.

2.~Ряд местоимений в позиции конъюнктов-подлежащих предписывают
согласование с ближайшим подлежащим. Сюда относятся вопросительные
местоимения \textit{кто} и \textit{что}, отрицательные \textit{никто} и
\textit{ничто}:

\begin{enumerate}
  \def\labelenumi{(\arabic{enumi})}
  \setcounter{enumi}{86}
  \item
        Как знать, \textit{кто} и \textit{что} \textit{окажется}
        \textless*\textit{окажутся}\textgreater{} итогом «переработки» в данном
        случае. {[}«Эксперт» (2004){]}
\end{enumerate}

Ср., впрочем, единственный в Корпусе пример с подлежащим \textit{кто} и
\textit{что} и сказуемым во множественном числе:

\begin{enumerate}
  \def\labelenumi{(\arabic{enumi})}
  \setcounter{enumi}{87}
  \item
        \textless\ldots\textgreater да и вообще не забыть бы имени при
        рассказах о том, \textit{кто} и \textit{что} \textit{были} причиной
        мученичества «нашего Ардальона Михайловича». {[}В.~В.~Крестовский.
        Панургово стадо (Ч. 3-4) (1869){]}
\end{enumerate}

При обратном порядке вопросительных местоимений --- подлежащем \textit{что
  и кто} --- с множественным числом сказуемого также обнаруживается
единственный пример (впрочем, данный линейный порядок и сам по себе
редкий --- 34~вхождения в Основном корпусе против 133~вхождений при
прямом порядке):

\begin{enumerate}
  \def\labelenumi{(\arabic{enumi})}
  \setcounter{enumi}{88}
  \item
        Вот я и объяснил, \textit{что и кто виноваты} сейчас. {[}коллективный.
        Социализм vs. Капитализм (2011){]}
\end{enumerate}

Согласно {[}Санников~2008:~153{]}, единственное число сказуемого
навязывают, кроме того, субстантивированные причастия или прилагательные
среднего рода в позиции сочиненных подлежащих, ср. \textit{хорошее и
  плохое смешалось \textless{}\textsuperscript{?}-лись\textgreater{} в
  памяти}. Хотя в Корпусе соответствующие примеры с множественным числом
сказуемого находятся, подавляющая их часть (ср., однако, (92)) содержит
в качестве сказуемого так называемый симметричный предикат
(\textit{переплетались} в (90)) или симметричную конструкцию
(\textit{обрабатываются отдельно} в (91)), которые выражают взаимную
соотнесенность конъюнктов друг с другом и поэтому предрасполагают к
множественному числу в силу своей семантики:

\begin{enumerate}
  \def\labelenumi{(\arabic{enumi})}
  \setcounter{enumi}{89}
  \item
        В решениях же Ивана Грозного \textit{закономерное} и
        \textit{парадоксальное} \textit{переплетались} так тесно, что, анализируя
        то и другое, поневоле задаешься вопросом: а как бы в этой ситуации
        поступил другой, более предсказуемый, «нормальный» политик? {[}«Вокруг
        света» (2004){]}
  \item
        \textit{Светлое} и \textit{темн}ое \textit{обрабатываются} отдельно.
        {[}Б.~Кенжеев. Из Книги счастья (2007){]}
  \item
        \textit{«Черное и белое»} \textit{делали} свое дело, мир упрощался,
        распадаясь на два цвета, уподобляясь телевизионному старому
        до-цветному фильму добрых шестидесятых, мир иллюзий. {[}В.~Аксенов.
        Остров Крым (1977-1979){]}
\end{enumerate}

Впрочем, наличие симметричного предиката в данном случае не обязывает к
множественному числу сказуемого (см.~также \underline{п.5}):

\begin{enumerate}
  \def\labelenumi{(\arabic{enumi})}
  \setcounter{enumi}{92}
  \item
        \textit{Благородное, жестокое и мстительное} \textit{сошлось} и врезалось
        на лице этой решительной калужской комсомолки, вовсе не красивой, в
        которой Кондрашёв-Иванов увидел Орлеанскую Деву! {[}А.~Солженицын. В
        круге первом (1968){]}
\end{enumerate}

Таким образом, причастия или прилагательные среднего рода в позиции
конъюнктов все-таки не являются бесспорно сильным фактором при выборе
модели согласования.

3.~Если сказуемое занимает позицию между конъюнктами, согласование
происходит с левым конъюнктом:

\begin{enumerate}
  \def\labelenumi{(\arabic{enumi})}
  \setcounter{enumi}{93}
  \item
        \textless\ldots\textgreater будто понимая, что и \textit{старость}
        \textit{наступит}, и \textit{болезни}. {[}С.~Есин. Марбург (2005){]}
\end{enumerate}

Это верно и тогда, когда левый конъюнкт отстоит от сказуемого дальше,
чем правый:

\begin{enumerate}
  \def\labelenumi{(\arabic{enumi})}
  \setcounter{enumi}{94}
  \item
        И \textit{семья} у них \textit{будет} \textless*-\textit{ут}\textgreater{} и
        \textit{дети} и, даже \textit{внуки}. {[}коллективный. Обсуждение фильма
        «Доживем до понедельника» (1968) (2007-2010){]}
\end{enumerate}

Таким образом, в этом (по-видимому, единственном) случае контролером
согласования может оказаться не ближайший к сказуемому конъюнкт. Это
связано, очевидно, с тем, что при расположении конъюнктов по разные
стороны от сказуемого тенденция к согласованию с ближайшим конъюнктом
пересиливается другой тенденцией, основанной на закономерностях процесса
речепорождения: согласовывать естественнее с тем, что уже произнесено,
чем с тем, что только последует (ср.~также ниже «слабый фактор~1»).

4.~Наличие при сочиненной группе определения в форме множественного
числа требует формы множественного числа сказуемого:

\begin{enumerate}
  \def\labelenumi{(\arabic{enumi})}
  \setcounter{enumi}{95}
  \item
        \textit{Приближались} \textless*\textit{приближался}\textgreater{}
        \textit{нечеловеческие} \textit{рев} и \textit{топот}. {[}М.~Елизаров.
        Библиотекарь (2007){]} --- cр. \textit{приближался рев и топот}
  \item
        Откуда им, бедным, знать, что "\textit{беспричинные богатство и
          знатность порождают} \textless*\textit{порождает}\textgreater{} смуту"?
        {[}«Завтра» (2003){]} --- ср.~\textit{богатство и знатность порождает}
\end{enumerate}

Данное требование связано, очевидно, с тем, что при согласовании с одним
и тем же существительным глагола и прилагательного желательно совпадение
их согласовательных стратегий.

5.~На выбор модели согласования влияет семантика сказуемого: форму
множественного числа имеет сказуемое, выраженное симметричным
предикатом, соотносящим субъекты друг с другом. Ср.:

\begin{enumerate}
  \def\labelenumi{(\arabic{enumi})}
  \setcounter{enumi}{97}
  \item
        Нельзя, чтобы \textit{бизнес} и \textit{власть} \textit{переплетались}
        \textless{}\textit{*переплеталась}\textgreater. {[}Б.~Немцов. Провинциал
        в Москве (1999){]}
  \item
        Когда мы скрепили договор рукопожатием, \textit{тьма} и \textit{свет}
        \textit{встретились} \textless{}\textit{*встретился}\textgreater{} между
        наших ладоней. {[}С.~Лукьяненко. Ночной дозор (1998){]}
\end{enumerate}

Исключение составляет случай, когда в позиции подлежащего сочинены
причастия или прилагательные среднего рода: в этом случае множественное
число симметричного предиката хотя и предпочтительно, но не обязательно
(см.~выше~п.2).

К числу слабых факторов (т.е.~действующих не всегда и в зависимости от
других факторов) при выборе модели согласования относятся следующие:

1.~Препозиция сказуемого способствует его согласованию с ближайшим
подлежащим, постпозиция --- согласованию со всей сочиненной группой. Ср.:

\begin{enumerate}
  \def\labelenumi{(\arabic{enumi})}
  \setcounter{enumi}{99}
  \item
        И вот уже \textit{исчез} и \textit{актер}, и \textit{женщина}
        \textless\ldots\textgreater. {[}К.~И.~Чуковский. Осип Дымов.
        Солнцеворот. Содружество. (1905){]} --- \textsuperscript{?}*И
        \textit{женщина}, и \textit{актер} \textit{исчез}.
  \item
        И я помню, как \textit{возмутилась} \textit{семья} и \textit{духовенство}.
        {[}митрополит Антоний (Блум). О болезнях (1995){]} --
        \textsuperscript{??}\textit{Духовенство и семья} \textit{возмутилась}.
\end{enumerate}

Данная закономерность объясняется тем, что при постпозиции сказуемого
контролер согласования оказывается по отношению к сказуемому в
предтексте и Говорящий, строя согласуемую форму, уже знает, с чем именно
ее согласовывать. Если же сказуемое препозитивно, к моменту построения
формы сказуемого форма контролера согласования еще не построена. Поэтому
согласование не со всей сочиненной группой, а только с ближайшим
конъюнктом более вероятно во втором случае, чем в первом.

2.~На выбор модели согласования влияет семантика союза: соединительный
\textit{и} способствует множественному числу сказуемого более, чем двойные
соединительные союзы, указывающие на неравноценность компонентов
(\textit{не~только\ldots но~и}, \textit{не~столько\ldots сколько} и~др., см.
\underline{Сочинительные союзы}) и более, чем разделительные союзы
\textit{или}, \textit{то~ли\ldots то~ли} и~др. Двойные союзы
\textit{не\ldots а} \textless{}\textit{а не}\textgreater{} и
\textit{не\ldots но}, за редким исключением, даже требуют единственного
числа сказуемого.

\begin{enumerate}
  \def\labelenumi{(\arabic{enumi})}
  \setcounter{enumi}{101}
  \item
        Не только орден, но и повышение \textit{светило}
        \textless{}\textsuperscript{?}\textit{светили}\textgreater{} майору.
        {[}М.~Гиголашвили. Чертово колесо (2007){]}
  \item
        Случай или закон \textit{определил}
        \textless*\textit{определили}\textgreater{} их судьбу? {[}В.~Гроссман.
        Все течет (1955-1963){]}
  \item
        Не он, а она \textit{смутилась} \textless*\textit{смутились}\textgreater{}
        (Л.~Толстой)
  \item
        Томск, утяжеленный гирей столичности, просто расколет страну: \textit{не
          вес}, \textit{но рычаг будет} \textless{}\textit{*будут}\textgreater{}
        \textit{слишком велик}\textless{}\textit{*-и}\textgreater. {[}Н.~Замятина.
        Места Нестолиц. Где нельзя размещать столицу (2003){]}
\end{enumerate}

Это различие имеет семантическую причину: соединительный \textit{и}
выражает, что все ситуации, обозначаемые конъюнктами, соответствуют
действительности, тогда как остальные из названных союзов в той или иной
мере подчеркивают соответствие действительности какой-то одной из
ситуаций. Союзы \textit{не\ldots а} \textless{}\textit{а~не}\textgreater{} и
\textit{не\ldots но} указывают, что только одна из двух соединяемых союзом
частей соответствует действительности, поэтому предпочтительность
единственного числа для них самая высокая.

Семантика союза, однако, не предопределяет окончательно модели
согласования: при соединительном и возможно единственное число
сказуемого, а при разделительном \textit{или} --- множественное (поэтому
данный фактор и является слабым). Ср.:

\begin{enumerate}
  \def\labelenumi{(\arabic{enumi})}
  \setcounter{enumi}{105}
  \item
        Его мучит рефлексия и ирония ― эти страсти развитого ума.
        {[}И.~Золотусский. «Записки сумасшедшего» и «Записки из подполья»
        (2002){]}
  \item
        \textless\ldots\textgreater{}\textit{не только душа}, \textit{но и ум
          нуждаются} в нравственной защищённости. {[}«Лебедь» (2003){]}
  \item
        Сейчас планируем (ему 13), чтобы \textit{папа или отчим поговорили} с
        ним подробнее. {[}Наши дети: Подростки (2004){]}
\end{enumerate}

Находятся в Корпусе и примеры с множественным числом сказуемого при
союзе \textit{не\ldots а} (хотя в {[}Санников~2008:~153{]} наличие данного
союза в сочинительной конструкции признается «сильным» фактором). Ср.:

\begin{enumerate}
  \def\labelenumi{(\arabic{enumi})}
  \setcounter{enumi}{108}
  \item
        Заметь, мой дорогой, \textit{не труд}, \textit{а смекалка сделали} из
        обезьяны человека. {[}С.~Романов. Парламент (2000){]}
  \item
        Но \textit{не месть}, \textit{а милосердие царили} в культурных
        государствах, и, почтительно простояв положенное время, даже мы
        получили визы. {[}И.~Г.~Эренбург. Необычайные похождения Хулио
        Хуренито (1921){]}
\end{enumerate}

Подробнее о связи модели согласования с семантикой союза
см.~{[}Санников~2008:~155--157{]}.

3.~Повторяющиеся союзы (\textit{и\ldots и}, \textit{ни\ldots ни} и~др.)
способствуют согласованию с ближайшим подлежащим более, чем одиночные.
Ср.~следующие контрастирующие примеры:

\begin{enumerate}
  \def\labelenumi{(\arabic{enumi})}
  \setcounter{enumi}{110}
  \item
        а.~\textit{Записался} и \textit{Низвецкий}, и \textit{сестра Фаина}, и даже
        \textit{Сухоедов}: он умел играть на балалайке. {[}В.~Ф.~Панова.
        Спутники (1945){]} --- б.\textsuperscript{?}\textit{Записался Низвецкий}
        и \textit{сестра Фаина}
  \item
        а.~\textsuperscript{?}Заходил и Коля, и Петя --- б.*Заходил Коля и Петя
\end{enumerate}

Данное различие связано с общими особенностями повторяющихся союзов. С
одной стороны, повторяющийся союз подчеркивает участие в сочинительном
отношении всех конъюнктов (подробнее см.~\underline{Сочинительные
  союзы}), что в определенном контексте способствует их интерпретации как
цельной единицы, а это, в свою очередь, согласуется с семантикой
единственного числа. Неслучайно при повторяющемся \textit{и\ldots и}
стандартно использование местоимения \textit{всё}, ср.:

\begin{enumerate}
  \def\labelenumi{(\arabic{enumi})}
  \setcounter{enumi}{112}
  \item
        И мы с тобой, и крысы, и дрозофилы ― \textit{всё едино}. {[}Л.~Улицкая.
        Казус Кукоцкого (2000){]}
\end{enumerate}

С другой стороны, при повторяющемся союзе сочиненные подлежащие
соотносятся со сказуемым по отдельности; это проявляется, в частности, в
том, что сказуемым не может быть симметричный предикат (нельзя *\textit{И
  Катя и Маша похожи}, так как невозможно *\textit{И Катя похожа}, \textit{и
  Маша похожа}). Такая отдельная интерпретация подлежащих тоже согласуется
с единственным числом сказуемого.

Поскольку ни одна из двух названных особенностей не свойственна
одиночному союзу, одиночный союз в меньшей степени, чем повторяющийся,
располагает к единственному числу сказуемого при сочиненном подлежащем.

Приоритет повторяющихся союзов, по-видимому, в первую очередь
сказывается при конъюнктах, выраженных одушевленными существительными,
поскольку неодушевленность конъюнктов сама по себе, независимо от типа
союза, является фактором, способствующим единственному числу сказуемого
(см. ниже \underline{фактор~5}). Ср. контрастирующие примеры
(111a)--(111б) и (112а)--(112б), где конъюнкты одушевленные, с почти
полным отсутствием контраста в примерах ниже, где конъюнкты
неодушевленные:

\begin{enumerate}
  \def\labelenumi{(\arabic{enumi})}
  \setcounter{enumi}{113}
  \item
        а.~Мне хотелось тоже попить чайку, у меня и \textit{чай}, и \textit{сахар}
        \textit{был} \textless\ldots\textgreater. {[}Ф.~М.~Решетников. Очерки
        обозной жизни (1867){]} --- б.~Мне хотелось тоже попить чайку, у меня
        \textit{чай} и \textit{сахар} \textit{был}.
\end{enumerate}

4.~Возможность согласования сказуемого с ближайшим подлежащим
облегчается совпадением характеристик рода и числа у всех сочиненных
подлежащих, поскольку в этом случае сказуемое оказывается потенциально
согласующимся с каждым из подлежащих. При этом, если подлежащие все-таки
различаются характеристиками рода, согласованию с ближайшим подлежащим
способствует отсутствие у сказуемого характеристики рода. Ср.~(115),
(116) и (117):

\begin{enumerate}
  \def\labelenumi{(\arabic{enumi})}
  \setcounter{enumi}{114}
  \item
        И \textit{смех}, и \textit{ужас напал} на меня, ― хлад мраза тонка, как
        говорят мистики. {[}Д.~С.~Мережковский. Александр первый (1922){]}
\end{enumerate}

В (115) у конъюнктов совпадают характеристики числа и рода, у сказуемого
в прошедшем времени наличествует характеристика рода --- предложение
приемлемо.

\begin{enumerate}
  \def\labelenumi{(\arabic{enumi})}
  \setcounter{enumi}{115}
  \item
        \textsuperscript{?}*И \textit{изумление}, и \textit{ужас} \textit{напал} на
        меня.
\end{enumerate}

В (116) у конъюнктов различаются характеристики рода, у сказуемого
наличествует характеристика рода --- предложение сомнительно.

\begin{enumerate}
  \def\labelenumi{(\arabic{enumi})}
  \setcounter{enumi}{116}
  \item
        И изумление, и ужас нападает на меня всякий раз при виде него.
\end{enumerate}

В (117) у конъюнктов различаются характеристики рода, у сказуемого в
настоящем времени отсутствует характеристика рода --- предложение
приемлемо.

5.~На выбор модели согласования влияет семантика подлежащего. По
способности индуцировать единственное число сказуемого существительные
выстраиваются в следующую иерархию (чем выше по иерархии, тем выше
вероятность единственного числа):

\textsc{имена собственные~\textless{} одушевленные
  нарицательные~\textless{} неодушевленные конкретные~\textless{}
  неодушевленные абстрактные}

Ср.~пример из {[}Санников~2008:~157{]}:

\begin{enumerate}
  \def\labelenumi{(\arabic{enumi})}
  \setcounter{enumi}{117}
  \item
        Во всем был \textit{виден точный расчет и удивительная
          целеустремленность} (единственное число сказуемого предпочтительно)
  \item
        Отсюда мне \textit{виден} \textless{}\textit{видны}\textgreater{}
        \textit{дом} и \textit{опушка} леса --- допустимы обе граммемы числа
  \item
        *Отсюда мне \textit{видны} \textless{}\textit{*виден}\textgreater{}
        \textit{Коля} и \textit{Маша} --- единственное число сказуемого недопустимо
\end{enumerate}

Данная закономерность дополняется, кроме того, тем, что среди
одушевленных нарицательных конъюнктов единственное число глагола
допускают, прежде всего, нереферентные именные группы, поскольку они
семантически сближаются с абстрактными существительными. Напротив,
референтные именные группы (обладающие признаком одушевленности)
семантически близки к именам собственным, поэтому они препятствуют
единственному числу сказуемого. Ср.:

\begin{enumerate}
  \def\labelenumi{(\arabic{enumi})}
  \setcounter{enumi}{120}
  \item
        *\textsuperscript{?}Сам постыдный плотский грех никак не назывался ― и
        \textit{прокурор}, и \textit{адвокат}, и \textit{свидетель вынужден был}
        говорить обиняками. (референтные конъюнкты)
  \item
        Бывает, что и вор, и преступник постучится в нашу
        дверь\textless\ldots\textgreater. {[}митрополит Антоний (Блум).
        Исцеление гергесинских бесноватых (1981){]} --- нереферентные конъюнкты
\end{enumerate}

Указанная упорядоченность существительных по семантическому типу
объясняется следующим образом. Множественное число сказуемого
предполагает интерпретацию сочиненных подлежащих как счетного множества
индивидуализированных объектов. С такой интерпретацией наилучшим образом
совместимы имена собственные (поскольку они обозначают конкретные
отдельные объекты) и наихудшим образом --- существительные абстрактной
семантики (обозначающие абстрактные сущности, не имеющие ясных границ и
очертаний).

\paragraph{Согласование в
  лице}\label{ux441ux43eux433ux43bux430ux441ux43eux432ux430ux43dux438ux435-ux432-ux43bux438ux446ux435}

Согласование сказуемого с сочиненным подлежащим в лице вычисляется
исходя из приоритета 1-го лица перед 2-м и 3-м и 2-го лица перед 3-м
лицом {[}Санников~2008:~160{]}, см.~также {[}Грамматика~1980{]},
{[}Munn~1999~и~др.{]}:

\textsc{1-е~л.~\textgreater~2-е~л.~\textgreater\textgreater~3-е~л.}

Ср.: \textit{Остаемся ты и я} \textless{}\textit{вы и я}; \textit{ты и мы};
\textit{я и они}\textgreater; \textit{остаетесь ты и они}
\textless{}\textit{вы и он и т}.\textit{д}. \textgreater.

Данный принцип приоритета отражает представление об относительной
важности участников речевого акта:

\textsc{Говорящий~\textgreater{}
  Слушающий~\textgreater\textgreater~третьи~лица}

Правила согласования в лице взаимодействуют с правилами согласования в
числе (см.~\underline{п.6.3.1.1}): итоговая форма сказуемого выбирается
с учетом суммарного действия того и другого. Так, в следующем примере
выбор формы сказуемого зависит от взаимного расположения подлежащего и
сказуемого: при постпозиции сказуемого недопустима, а при препозиции
допустима форма 2-го~л.~ед.ч. \textit{остаешься} (несмотря на то, что эта
форма в обоих случаях согласуется с вышеуказанным принципом приоритета):

\begin{enumerate}
  \def\labelenumi{(\arabic{enumi})}
  \setcounter{enumi}{122}
  \item
        а. И учителя старших классов, и ты \textit{остаетесь}
        \textless*\textit{остаешься}\textgreater. --- б.~\textit{Остаешься}
        \textless{}\textsuperscript{?}\textit{остаетесь}\textgreater{} ты и
        учителя старших классов.
\end{enumerate}

Это объясняется тем, что форма \textit{остаешься} отражает согласование
сказуемого по числу не со всей сочиненной группой, а с ближайшим
конъюнктом; такое скорее возможно при препозиции сказуемого, чем при
постпозиции (см.~\underline{п.6.3.1.1}). Таким образом, принципом
приоритета «отбираются» допустимые формы 2-го лица в обоих числах, из
которых предпочтительная числовая форма определяется исходя из фактора
линейного порядка, релевантного для согласования в числе.

Кроме того, само по себе согласование в лице подчиняется действию
большинства тех факторов, которые играют роль при согласовании в числе
(см.~\underline{п.6.3.1.1}). Если действует какой-либо фактор,
индуцирующий согласование с ближайшим конъюнктом, указанное выше правило
приоритета может нарушаться. Так, данное правило часто нарушается в
предложениях с союзом, указывающим на неравноценность компонентов
(\textit{не только\ldots но~и}, \textit{не столько\ldots сколько} и~др.).
Ср.:

\begin{enumerate}
  \def\labelenumi{(\arabic{enumi})}
  \setcounter{enumi}{123}
  \item
        Не только я, но и другие не сразу \textit{понимают}
        \textless{}\textsuperscript{?}\textit{понимаем}\textgreater, что ты
        говоришь. (Яндекс)
  \item
        Но гораздо лучше, когда не столько ты, сколько другие \textit{считают}
        \textless*\textit{считаем}\textgreater{} тебя красивой. (Яндекс)
\end{enumerate}

Аналогично, нарушение правила приоритета скорее возможно при препозиции
сказуемого, чем при его постпозиции. Ср:

\begin{enumerate}
  \def\labelenumi{(\arabic{enumi})}
  \setcounter{enumi}{125}
  \item
        а.~*Ты и учителя старших классов \textit{остаются}. --
        б.~\textit{Остаются} учителя старших классов и ты.
\end{enumerate}

В целом, однако, условия нарушения общего принципа приоритета до конца
не ясны. По-видимому, его нарушению может способствовать, в частности,
значительная семантическая разнородность конъюнктов, затрудняющая их
объединение в цельное множество, ср.~пример из {[}Санников~2008:~161{]}:

\begin{enumerate}
  \def\labelenumi{(\arabic{enumi})}
  \setcounter{enumi}{126}
  \item
        Его \textit{погубит}
        \textless*\textsuperscript{?}\textit{погубите}\textgreater{} любовь к
        выпивке и ты.
\end{enumerate}

\subsubsection{Согласование прилагательного с сочиненными
  существительными в
  числе}\label{ux441ux43eux433ux43bux430ux441ux43eux432ux430ux43dux438ux435-ux43fux440ux438ux43bux430ux433ux430ux442ux435ux43bux44cux43dux43eux433ux43e-ux441-ux441ux43eux447ux438ux43dux435ux43dux43dux44bux43cux438-ux441ux443ux449ux435ux441ux442ux432ux438ux442ux435ux43bux44cux43dux44bux43cux438-ux432-ux447ux438ux441ux43bux435}

Выбор числа прилагательного в конструкции типа \textit{модная}
\textless{}\textit{-ые}\textgreater{} \textit{шляпа и галстук} регулируется
следующими основными факторами {[}Кодзасов~1987:~213--219{]}:

1.~Данная конструкция может быть омонимична с точки зрения того, к чему
относится прилагательное: к первому конъюнкту (несимметричная атрибуция)
или ко всем конъюнктам (симметричная атрибуция). Неоднозначность при
омонимии снимается формой множественного числа прилагательного. Ср.
двузначность сочетания \textit{модная шляпа и галстук} (`модная шляпа и
модный галстук' или `модная шляпа и галстук') и однозначность сочетания
\textit{модные шляпа и галстук}\footnote{В устной речи омонимия, как
  правило, снимается интонацией: при отнесенности прилагательного только
  к первому конъюнкту этот конъюнкт получает коммуникативно релевантный
  акцент (\textit{модная шляпа/ и галстук\textbackslash{}}), тогда как при
  отнесенности прилагательного ко всей сочиненной группе
  акцентоносителем обычно бывает только заключительный конъюнкт
  (\textit{модная шляпа и галстук~/} или \textit{модная/ шляпа и
    галстук\textbackslash{}}).}.

2.~При симметричной атрибуции множественное и единственное число
прилагательного различаются семантически --- с точки зрения
цельности~/~распределенности атрибуции. При цельной атрибуции,
т.е.~множественном числе прилагательного, предполагается внутренняя
связанность определяемых объектов; при распределенной атрибуции, т.е.
единственном числе прилагательного, --- их отдельное существование. Во
многих случаях данная семантическая разница между цельной и
распределенной атрибуцией едва уловима, ср. \textit{алюминиевая миска и
  кружка} (объекты подвергаются атрибуции порознь) vs. \textit{алюминиевые
  миска и кружка} (объекты подвергаются атрибуции вместе). Однако различие
становится существенным, если обозначаемые объекты воспринимаются как
единое целое: в этом случае цельная атрибуция и, соответственно,
множественное число атрибута обязательны, ср.: \textit{будущие муж и
  жена}, \textit{прежние Босния и Герцеговина}, \textit{любящие брат и
  сестра}.

3.~При симметричной атрибуции число прилагательного зависит от
семантического класса существительного; вероятность единственного числа
вырастает в соответствии со следующей иерархией (аналогичный фактор
действует при согласовании сказуемого с сочиненным подлежащим):

\textsc{имена собственные~\textless{}
  одушевленные~нарицательные~\textless{}
  неодушевленные~конкретные~\textless{} неодушевленные~абстрактные}

Ср.:

\begin{enumerate}
  \def\labelenumi{(\arabic{enumi})}
  \setcounter{enumi}{127}
  \item
        Мы считаем, что роль и значение ООН, как универсального инструмента по
        поддержанию \textit{международного}
        \textless{}\textit{\textsuperscript{?}-ых}\textgreater{} \textit{мира} и
        \textit{безопасности}, должна быть укреплена. {[}«Дипломатический
        вестник» (2004){]}
  \item
        В октябре 2002 года в Англии из дерева, птичьих перьев, выделений
        жуков были изготовлены лётные приспособления, разработанные Леонардо,
        ― прообразы \textit{современного} \textless{}\textit{-ых}\textgreater{}
        \textit{парашюта} и \textit{дельтаплана}. {[}«Известия» (2002){]}
  \item
        Она не была истовой коммунисткой, почитала \textit{верующих}
        \textless{}\textit{\textsuperscript{?}-его}\textgreater{} \textit{отца и
          мать}\textless\ldots\textgreater. {[}В.~Астафьев. Затеси (1999){]}
  \item
        Есть \textit{перспективные}
        \textless{}\textit{\textsuperscript{?}-ый}\textgreater{} \textit{Пименов и
          Измайлов}, которые уже числятся в кандидатах в сборную, а наши игроки
        в два раза менее перспективны. {[}«Известия» (2002){]}
\end{enumerate}

Основная причина данной закономерности состоит в том, что абстрактные
имена обычно грамматически и семантически неисчисляемы. Поэтому их
объединение в множество индивидуализированных объектов, имеющее место
при множественном числе атрибута, затруднено. Наоборот, семантика имен
собственных такому объединению способствует.

Указанная закономерность не является строгой. В Корпусе находится как
форма ед.ч. прилагательного при одушевленных конъюнктах, так и форма
мн.ч. при неодушевленных конъюнктах. Ср.:

\begin{enumerate}
  \def\labelenumi{(\arabic{enumi})}
  \setcounter{enumi}{131}
  \item
        Но встретились они лишь час спустя, поочерёдно отобщавшись со
        \textit{словенским президентом} и \textit{премьером}. {[}«Известия»
        (2001){]}
  \item
        И тогда все снова будут вслух поносить \textit{германские
          прямолинейность} и \textit{бездушие} да жалеть чувственные команды с
        Пиреней. {[}«Известия» (2002){]}
\end{enumerate}

4.~Особым образом ведут себя указательные и кванторные местоимения
(см.~\underline{Местоимения}) в позиции атрибута. Указательные
местоимения чаще выступают в форме множественного числа, независимо от
семантического типа сочиняемых существительных. Ср.:

\begin{enumerate}
  \def\labelenumi{(\arabic{enumi})}
  \setcounter{enumi}{133}
  \item
        Как пролавировал я между \textit{этими}
        \textless{}\textit{\textsuperscript{?}-ой}\textgreater{} \textit{Сциллой и
          Харибдой}, предоставляю на суд читателя. {[}А.~А.~Фет. Осенние хлопоты
        (1862){]}
  \item
        \textless\ldots\textgreater но, с другой стороны, неприкаянность эта
        была как бы только внешней, иначе откуда \textit{эти беспечность и
          свобода}, которые никогда не были ему свойственны? {[}А.~Битов.
        Рассеянный свет (1981){]}
\end{enumerate}

Это связано с тем, что при распределенной атрибуции, навязываемой
единственным числом атрибута, желателен повтор указательного местоимения
при неначальном конъюнкте: этот погреб и этот сарай; эта беспечность и
эта свобода. Причина этого, в свою очередь, в том, что референция
указательного местоимения уникальна, поэтому местоимения обычно не могут
быть отождествлены друг с другом. Между тем в отсутствие повтора
естественно считать, что местоимение при неначальном конъюнкте
эллиптировано, т.е.~как раз отождествлено с наличествующим местоимением,
ср.~\textit{этот погреб} и \textless подразумевается:
\textit{этот}\textgreater{} \textit{сарай}.

Кванторные местоимения (\textit{весь}, \textit{каждый} и~т.п.), наоборот,
чаще выступают в форме единственного числа, в том числе и при именах
собственных:

\begin{enumerate}
  \def\labelenumi{(\arabic{enumi})}
  \setcounter{enumi}{135}
  \item
        \textless..\textgreater он исколесил \textit{всю}
        \textless*\textit{все}\textgreater{} \textit{Британию} и \textit{Европу}, ни
        на минуту не расставаясь с карандашом и блокнотом. {[}«Наука и жизнь»
        (2009){]}
  \item
        Она заполонила собой всё ― эфир, печать, рекламные щиты и скоро, надо
        полагать, будет глядеть на нас с \textit{каждого}
        \textless*\textit{каждых}\textgreater{} \textit{столба} и \textit{забора}.
        {[}«Завтра» (2003){]}
\end{enumerate}

Исключение составляют случаи, когда конъюнкты по некоторому основанию
интерпретируются как единое целое, как цельная пара, что располагает к
цельной атрибуции и, соответственно, множественному числу атрибута
(см.~выше \underline{пункт~2}). Ср.:

\begin{enumerate}
  \def\labelenumi{(\arabic{enumi})}
  \setcounter{enumi}{137}
  \item
        Такие приемы ведутся регулярно, \textit{каждые понедельник} и
        \textit{четверг}. {[}«Встреча» (2003){]}
  \item
        Каждый листик, каждые зверь и зверюшка, \textit{каждые} \textit{жуки} и
        \textit{мушки} являлись для него и для детей предметами жадного внимания
        и удивительных рассказов. {[}А.~И.~Куприн. Жанета (1933){]}
  \item
        Из Столицы нашей родины скрыться в данном марокканском направленности
        возможно станет \textit{всякие} \textit{вторник} и \textit{пятницу} (Яндекс)
\end{enumerate}

Предрасположенность кванторных атрибутов к форме ед.ч. объясняется
семантической нерегулярностью их числовых форм
{[}Зализняк,~Падучева~1974{]}.

5.~Единственное число прилагательного обязательно при денотативном
тождестве конъюнктов:

\begin{enumerate}
  \def\labelenumi{(\arabic{enumi})}
  \setcounter{enumi}{140}
  \item
        В понедельник в Москве хоронили \textit{известного}
        \textless*-\textit{ых}\textgreater{} \textit{программиста} и
        \textit{журналиста} \textit{Владимира Сухомлина}. {[}«Известия» (2003){]}
\end{enumerate}

6.~Единственное число прилагательного вероятнее при совпадении рода у
существительных. Ср. \textit{красивая шляпа и блузка} vs.
\textsuperscript{?}\textit{красивая} \textless{}\textit{-ые}\textgreater{}
\textit{шляпа и платье}.

7.~Единственному числу конъюнктов способствует разделительная семантика
сочинительного союза:

\begin{enumerate}
  \def\labelenumi{(\arabic{enumi})}
  \setcounter{enumi}{141}
  \item
        Достаточно приварить к опоре железный \textless?-ые\textgreater{}
        стержень или трубу ― и молоток готов. {[}«Народное творчество»
        (2004){]}
\end{enumerate}

Подробнее о согласовании прилагательного с сочиненными существительными
см.~{[}Кодзасов~1987:~213--219{]}.

\subsubsection{Согласование в числе в группе с сочиненными
  прилагательными}\label{ux441ux43eux433ux43bux430ux441ux43eux432ux430ux43dux438ux435-ux432-ux447ux438ux441ux43bux435-ux432-ux433ux440ux443ux43fux43fux435-ux441-ux441ux43eux447ux438ux43dux435ux43dux43dux44bux43cux438-ux43fux440ux438ux43bux430ux433ux430ux442ux435ux43bux44cux43dux44bux43cux438}

Данный случай отличается от двух рассмотренных выше
(см.~\underline{п.6.3.1}, \underline{п.6.3.2}) тем, что граммема числа
варьируется не у прилагательного, обычно выступающего мишенью
согласования в атрибутивной конструкции, а у существительного --
прототипического контролера, ср.~\textit{белая и красная роза}
\textless{}\textit{розы}\textgreater. Причем число существительного
выбирается вне прямой связи с числом прилагательного (а исходя из ряда
преимущественно семантических факторов, см.~ниже), поэтому говорить о
собственно согласовании здесь можно лишь условно.

Колебания в числовой форме существительного возможны, как правило,
тогда, когда оба прилагательных имеют форму единственного числа; редкие
исключения составляют конструкции, в которых левое прилагательное
получает семантически мотивированную форму мн.ч., ср.:

\begin{enumerate}
  \def\labelenumi{(\arabic{enumi})}
  \setcounter{enumi}{142}
  \item
        Уже созданы основные ветви судопроизводства― суды общей юрисдикции,
        \textit{арбитражные} и Конституционный суд. {[}«Итоги» (2003){]}
\end{enumerate}

Ниже рассмотрен только прототипический случай: колебания в форме
существительного при единственном числе обоих сочиненных прилагательных.

Выбор числа существительного в данной конструкции регулируется
следующими основными факторами {[}Кодзасов~1987:~205--211{]}.

1.~Количество референтов

1.1.~Колебания в числе существительного возможны лишь в том случае,
когда обсуждаемая конструкция обозначает более одного объекта,
ср.~\textit{У него} \textit{красивая и умная жена}
\textless{}\textit{*жены}\textgreater.

1.2.~В случае, когда конструкция может обозначать как один, так и более
одного объекта, т.е.~допускает омонимию, средством для устранения
неоднозначности служит множественное число существительного. Так,
сочетание \textit{твой и мой дом} имеет два значения: `твой и мой дом --
один и тот же дом' и `твой дом и мой дом --- два разных дома'. Форма
множественного числа снимает неоднозначность, ср. \textit{твой и мой
  дома}.

2.~Числовые свойства имени

2.1.~(Не)исчисляемость имени

Неисчисляемость существительного предопределяет форму единственного
числа: \textit{слуховая и зрительная память}, \textit{качественный и
  количественный анализ}, \textit{экспериментальная и клиническая хирургия}.
Если лексема имеет исчисляемое и неисчисляемое значения, ее
использование в неисчисляемом значении требует формы единственного
числа:

\begin{enumerate}
  \def\labelenumi{(\arabic{enumi})}
  \setcounter{enumi}{143}
  \item
        Может использоваться для имитации ореха, \textit{красного и чёрного
          дерева} \textless*\textit{деревьев}\textgreater{} в мозаичных наборах.
        {[}Т.~Матвеева. Реставрация столярно-мебельных изделий (1988){]}
\end{enumerate}

vs.

\begin{enumerate}
  \def\labelenumi{(\arabic{enumi})}
  \setcounter{enumi}{144}
  \item
        В толковании мачехи --- \textit{старое и молодое деревья} являются
        прообразами царя и его сына (Яндекс)
\end{enumerate}

2.2.~Семантическая специфика формы множественного числа имени

Некоторые классы объектов имеют множественное число нестандартной
семантики. Сюда относится: «собирательное» множественное (\textit{волосы},
\textit{листья}, \textit{вещи}, \textit{люди} и~под.), «двойственное»
множественное (\textit{глаза}, \textit{руки}, \textit{ботинки}, \textit{берега}
и~под.), «взаимное» множественное (\textit{сестры}, \textit{коллеги},
\textit{друзья}), см.~\underline{Число}. Во всех таких случаях объекты,
обозначаемые формой множественного числа, прототипически мыслятся как
связанная совокупность. Между тем единственное число атрибутов в
обсуждаемой конструкции предполагает нейтральное --- разделительное --
понимание совокупности объектов. Поэтому множественное число имени
обычно избегается, и для обозначения нейтральной множественности
используется форма единственного числа. Ср.: \textit{красный и розовый
  цветок} \textless{}\textit{?цветы}\textgreater; \textit{здоровая и больная
  рука} \textless{}\textit{?руки}\textgreater; \textit{правый и левый сапог}
\textless{}\textsuperscript{?}\textit{сапоги}\textgreater; \textit{Его
  старый и новый друг} \textless{}\textit{?друзья}\textgreater{}
\textit{постоянно враждовали} (примеры из {[}Кодзасов~1987:~207--208{]}).

3) Выбор числа прилагательного зависит от денотативного статуса именной
группы (о денотативном статусе см. статью~\underline{Референциальный
  статус}, а также {[}Падучева~1985{]}). При родовом статусе группы
единственное число существительного предпочтительно в случае, если
сочиняемые понятия мыслятся интенсионально --- как эталонная сущность,
воплощающая типичные свойства класса. Ср.:

\begin{enumerate}
  \def\labelenumi{(\arabic{enumi})}
  \setcounter{enumi}{145}
  \item
        \textit{Африканский и индийский слон}
        \textless{}\textsuperscript{?}\textit{слоны}\textgreater{} почти
        полностью истреблены (пример из {[}Кодзасов 1987: 208{]})
  \item
        Водилась \textit{европейская и американская норка}
        \textless{}\textit{*норки}\textgreater, куница, хорек, енотовидная
        собака, было несколько больших колоний барсуков, очень много белки.
        {[}«Бельские Просторы» (2010){]}
  \item
        В области подготовки кадров особого внимания требуют два самых слабых
        звена. Одно из них ― \textit{начальная и средняя школа}
        \textless{}\textit{?школы}\textgreater, где уровень знаний выпускников
        пока неуклонно снижается. {[}«Лесное хозяйство» (2004){]}
\end{enumerate}

Ср. множественное число существительного при экстенсиональном
употреблении понятия --- как множества объектов данного класса:

\begin{enumerate}
  \def\labelenumi{(\arabic{enumi})}
  \setcounter{enumi}{148}
  \item
        Общая продолжительность полного курса среднего образования
        (\textit{включая начальную и профильную школы}) составляет 13 лет.
        {[}«Вопросы психологии» (2004){]}
\end{enumerate}

Здесь понятие \textit{школа} мыслится как множество, состоящее из
нескольких элементов, в числе которых \textit{начальная школа} и
\textit{профильная школа}.

Впрочем, граница между интенсиональным и экстенсиональным прочтением в
данной конструкции не является жесткой, поэтому не является строгим и
правило выбора конкретной числовой формы. Размытость границ между двумя
прочтениями связана с тем, что обсуждаемая конструкция допускает две
разновидности экстенсиональной интепретации: 1)~как множества,
состоящего из отдельных представителей класса и 2)~как множества,
состоящего из двух элементов, задаваемых прилагательными. Так, сочетание
\textit{африканский и индийский слоны} может пониматься как 1)~множество,
состоящее из представителей африканских и индийских слонов и 2)~как
множество, состоящее из двух элементов --- класса африканских слонов и
класса индийских слонов. Первое прочтение является прототипически
экстенсиональным; второе, будучи экстенсиональным, сближается, вместе с
тем, с интенсиональным, поскольку внутри двух элементов множества --
африканские и индийские слоны --- уже не различаются отдельные
представители. Таким образом, допустимость второй, пограничной
интерпретации делает во многих случаях возможными оба прочтения и,
соответственно, обе граммемы числа.

Так, в следующем примере слово \textit{китов} допускает обе указанные
разновидности экстенсионального прочтения: 1)~множество, состоящее из
представителей класса серых китов и представителей класса гренландских
китов и 2)~множество, состоящее из двух элементов: \textit{серые киты} и
\textit{гренландские киты}.

\begin{enumerate}
  \def\labelenumi{(\arabic{enumi})}
  \setcounter{enumi}{149}
  \item
        Охотятся на \textit{серого} и \textit{гренландского} \textit{китов}, на
        моржей и на лахтака. {[}В.~Писигин. Письма с Чукотки (2001){]}
\end{enumerate}

Второе прочтение близко к интенсиональной интерпретации, поэтому данный
контекст не исключает и формы единтсвенного числа существительного. Ср.:

\begin{enumerate}
  \def\labelenumi{(\arabic{enumi})}
  \setcounter{enumi}{150}
  \item
        Охотятся на \textit{серого} и \textit{гренландского} \textit{кита}, на
        моржей и на лахтака.
\end{enumerate}

Экстенсиональная интерпретация и множественное число существительного
строго запрещены, по-видимому, лишь в тех случаях, когда важно
подчеркнуть целостность сущности, воплощающей типичные свойства
класса\footnote{Речь идет о строгости обсуждаемого собственно
  семантического запрета; помимо него, разные другие факторы могут
  влиять на выбор числовой формы. Так, в примере \textit{Водилась
    европейская и американская норка}
  \textless{}\textit{*норки}\textgreater{} строгость запрета на
  множественное число существительного связана, по-видимому, с тем, что
  сказуемое имеет форму ед.ч. Ср.~допустимость мн.ч. существительного
  при мн.ч. сказуемого: \textit{Водились европейская и американская
    норки}.}. Ср.~(примеры заимствованы из {[}Кодзасов~1987:~208{]}):

\begin{enumerate}
  \def\labelenumi{(\arabic{enumi})}
  \setcounter{enumi}{151}
  \item
        \textit{Научная} и \textit{педагогическая} \textit{книга}
        \textless*\textit{книги}\textgreater{} все больше вытесняет
        беллетристику.
  \item
        В образовании и воспитании подрастающего поколения особенно велика
        роль \textit{средней} и \textit{высшей} \textit{школы}
        \textless*\textit{школ}\textgreater.
\end{enumerate}

4.~Единственному числу существительного способствует разделительная
семантика сочинительного союза:

\begin{enumerate}
  \def\labelenumi{(\arabic{enumi})}
  \setcounter{enumi}{153}
  \item
        В этом случае отправитель перевода должен открыть \textit{текущий} или
        \textit{депозитный} \textit{счёт} \textless*\textit{счета}\textgreater{} в
        банке. {[}«Вопросы статистики» (2004){]} --- \textit{текущий} и
        \textit{депозитный счет} \textless{}\textit{счета}\textgreater{} \textit{в
          банке}
\end{enumerate}

Данная закономерность действует, прежде всего, в тех случаях, когда
разделительный союз понимается в смысле строгой дизъюнкции (о
разграничении строгой и нестрогой дизъюнкции
см.~\underline{п.7.1.2.~Разделительность}). При нестрогой дизъюнкции
допустима и форма множественного числа существительного:

\begin{enumerate}
  \def\labelenumi{(\arabic{enumi})}
  \setcounter{enumi}{154}
  \item
        Для приготовления домашнего йогурта нам потребуются молоко и закваска
        ― «\textit{зелёный}» или \textit{цветной} \textit{йогурты}, которые не
        подвергались термической обработке после заквашивания. {[}«Наука и
        жизнь» (2009){]}
\end{enumerate}

В целом, выбор конкретной числовой формы существительного в конструкции
с сочиненными прилагательными, равно как и выбор числа прилагательного в
конструкции с сочиненными существительными (см.~\underline{п.6.3.2}),
выбирается на основе сложного взаимодействия перечисленных и некоторых
других факторов (морфологических, синтаксических, семантических,
коммуникативных и стилистических).

Подробнее о согласовании существительного с сочиненными прилагательными
см. {[}Кодзасов~1987:~205--213{]}.

\subsubsection{Согласование с участием сочиненной группы
  (итоги)}\label{ux441ux43eux433ux43bux430ux441ux43eux432ux430ux43dux438ux435-ux441-ux443ux447ux430ux441ux442ux438ux435ux43c-ux441ux43eux447ux438ux43dux435ux43dux43dux43eux439-ux433ux440ux443ux43fux43fux44b-ux438ux442ux43eux433ux438}

Три рассмотренных случая согласования по категории числа
(см.~\underline{п.6.3.1.1}, \underline{п.6.3.2} и \underline{п.6.3.3})
имеют как различия, так и сходства.

К сходствам относится наличие во всех случаях семантической мотивации,
стоящей за выбором конкретной числовой формы, --- количество референтов
описываемой ситуации.

Что касается других, более частных факторов, случай согласования
сказуемого с подлежащим (см.~\underline{п.6.3.1.1}) обнаруживает
сходство со случаем согласования прилагательного с сочиненными
существительными (см.~\underline{п.6.3.2}), поскольку в обоих случаях
сочиняются существительные. Третий же случай --- сочинение прилагательных
(см.~\underline{п.6.3.3}) --- стоит особняком. Так, общими для
согласования сказуемого с подлежащим и согласования прилагательного с
сочиненными существительными являются следующие факторы: денотативное
тождество конъюнктов; совпадение рода конъюнктов; семантический тип
существительных на шкале одушевленности. При сочинении прилагательных
данные факторы нерелевантны.

Вместе с тем, если иметь в виду другой --- формальный --- параметр
классификации, обсуждаемые три случая группируются иначе: случаи
согласования в атрибутивной конструкции (см.~\underline{п.6.3.2} и
\underline{п.6.3.3}) объединяются в одну группу, а согласование
сказуемого с подлежащим (см.~\underline{п.6.3.1.1}) оказывается в
стороне. Общим для случаев согласования в атрибутивной конструкции
является тот факт, что линейная позиция атрибутивного прилагательного
относительно существительного в русском языке фиксирована. Поэтому
здесь, в частности, не может реализоваться фактор, релевантный для
согласования сказуемого с подлежащим (см.~\underline{п.6.3.1.1}) --
взаимное расположение контролера и мишени согласования.

\section{Семантика
  сочинения}\label{ux441ux435ux43cux430ux43dux442ux438ux43aux430-ux441ux43eux447ux438ux43dux435ux43dux438ux44f}

\subsection{Основные семантические зоны
  сочинения}\label{ux43eux441ux43dux43eux432ux43dux44bux435-ux441ux435ux43cux430ux43dux442ux438ux447ux435ux441ux43aux438ux435-ux437ux43eux43dux44b-ux441ux43eux447ux438ux43dux435ux43dux438ux44f}

Для сочинительной конструкции в языках мира наиболее типичными являются
следующие значения: соединительное, разделительное, противительное.
Внутри каждой из этих трех семантических зон языки могут проводить
дальнейшее семантическое деление.

\subsubsection{Соединительность
  (конъюнкция)}\label{ux441ux43eux435ux434ux438ux43dux438ux442ux435ux43bux44cux43dux43eux441ux442ux44c-ux43aux43eux43dux44aux44eux43dux43aux446ux438ux44f}

Соединительная семантика выражается такими союзами, как рус. \textit{и},
англ. \textit{and}, нем. \textit{und} и~т.д.

В сочинительных сочетаниях, соответствующих семантической зоне
соединительности, собственно соединительное значение часто осложняется
рядом «асимметричных» значений --- временного следования, следствия,
противительности. Ср.:

\begin{enumerate}
  \def\labelenumi{(\arabic{enumi})}
  \setcounter{enumi}{155}
  \item
        Имею \textit{аттестат} о полном среднем образовании \textit{и} серебряную
        \textit{медаль}. {[}Автобиография (2006){]} --- собственно соединительное
        значение
  \item
        Я все \textit{ищу добра} --- \textit{и} \textit{нахожу} лишь \textit{зло}.
        (А.~Фет) --- противительное значение
\end{enumerate}

Дискуссионным является вопрос о том, входят ли данные асимметричные
значения в семантику собственно союза или образуются под воздействием
контекста и лексического состава конъюнктов. Согласно
{[}Санников~1989{]}, {[}Санников~2008{]}, {[}Урысон~2011{]}, значение
следствия выражается именно союзом и, а не контекстом. Так, в следующем
примере (158в) в выражена причинно-следственная связь между пропозициями
клауз, а в примерах (158а) и (158б), отличающихся от (158в) отсутствием
союза \textit{и}, такая связь не выражена:

\begin{enumerate}
  \def\labelenumi{(\arabic{enumi})}
  \setcounter{enumi}{157}
  \item
        а.~Коля ушел домой, а Петя остался в школе.
\end{enumerate}

б.~Коля ушел домой, Петя остался в школе.

в.~Коля ушел домой, и Петя остался в школе.

Ряд авторов придерживается противоположной точки зрения. Так, в
{[}Blakemore,~Carston~2005{]} обосновывается позиция, согласно которой
все асимметричные значения в конструкциях с союзом \textit{and} могут быть
выведены из взаимодействия чисто конъюнктивного значения \textit{and} с
контекстом. Такое взаимодействие регулируется прагматическими факторами,
в частности теми, которые описаны коммуникативными постулатами П.~Грайса
{[}Grice~1975{]}. Тем самым, у \textit{and} выделяется единственное --
соединительное --- значение.

Внутри семантической зоны чистой конъюнкции языки также могут проводить
семантическое дробление. Так, типологически частотным является
разграничение «естественной» конъюнкции (natural conjunction), члены
которой концептуально близки, и «случайной» конъюнкции (accidental
conjunction), объединяющей понятия случайным образом. Первый тип
иллюстрируют примеры \textit{мать и отец}, \textit{муж и жена}, \textit{лук и
  стрелы}, \textit{причина и следствие} и~т.п.; ко второму типу относятся
примеры \textit{Маша и Петя}, \textit{красный и синий}, \textit{стул и
  кровать} и другие сочетания, связанные с конкретной ситуацией, а не с
языковой картиной мира. При «естественной» конъюнкции, как правило,
фиксирован линейный порядок конъюнктов
(см.~\underline{п.4.1.~Необратимость при сочинении}). Различие между
«естественной» и «случайной» конъюнкцией может быть существенно при
определении сферы действия некоторых элементов, относящихся к обоим
конъюнктам. Так, в русском языке опущение второго предлога при сочинении
предложных групп вероятнее, если конъюнкты концептуально близки
{[}Пешковский~1928/2001:~448{]}, {[}Санников~2008:~140{]}. Ср.: \textit{с
  матерью и отцом}, \textit{с братом и сестрой}, \textit{о причинах и
  следствиях} vs.:

\begin{enumerate}
  \def\labelenumi{(\arabic{enumi})}
  \setcounter{enumi}{158}
  \item
        \textless\ldots\textgreater{} \textit{в горле и в груди}
        \textless{}\textsuperscript{?}\textit{в горле и груди} \textgreater{}
        становится тепло. {[}Е.~Гришковец. ОдноврЕмЕнно (2004){]}
  \item
        Говоря это, Юлия Михайловна подбоченивалась и смотрела на мужа и на
        Глебова \textless{}\textsuperscript{?}\textit{на мужа и
          Глебова}\textgreater{} несколько свысока. {[}Ю.~Трифонов. Дом на
        набережной (1976){]}
\end{enumerate}

В английском языке концептуальная близость конъюнктов разрешает опущение
определенного артикля при втором конъюнкте: \textit{the house and garden}
`дом и сад', но *\textit{the house and stamp collection} `дом и коллекция
марок' {[}Haspelmath~2007:~23{]}.

Особая, типологически распространенная разновидность конъюнкции --
«усилительная» конъюнкция (augmentative conjunction), при которой та или
иная усилительная семантика достигается повторением тождественных
словоформ, ср. \textit{шел и шел}, \textit{рос и рос} и~т.п. В русском языке
имеется несколько семантических разновидностей усилительной конъюнкции.
Основные из них следующие {[}Санников~2008:~365--378{]}:

1.~\textit{Он рос и рос}. Ср.:

\begin{enumerate}
  \def\labelenumi{(\arabic{enumi})}
  \setcounter{enumi}{160}
  \item
        Мозги не могли успокоиться и все \textit{думали и думали}, как жернова,
        перемалывающие что-то, непонятно что. {[}П.~Мейлахс. Избранник
        (1996){]}
  \item
        Так или иначе, а интерес к Туле с каждым веком \textit{рос и рос}.
        {[}А.~Балабуха. «Лев Массалии» (2009){]}
  \item
        Народ всё \textit{шёл и шёл} на Цветной к гробу своего любимца.
        {[}И.~Э.~Кио. Иллюзии без иллюзий (1995-1999){]}
\end{enumerate}

Конструкция выражает длительность и непрерывность действия. При повторе
глагола без участия союза данный смысл не обязательно выражен, ср.:

\begin{enumerate}
  \def\labelenumi{(\arabic{enumi})}
  \setcounter{enumi}{163}
  \item
        \textit{Терпи}, \textit{терпи}, ― ласково бормотала Пелагея Ивановна,
        наклоняясь к женщине, ― доктор сейчас тебе поможет\ldots{}
        {[}М.~А.~Булгаков. Крещение поворотом (1925){]} --- vs. *\textit{Терпи и
          терпи}. Доктор сейчас тебе поможет.
\end{enumerate}

2.~\textit{Снег да снег кругом}. Ср.:

\begin{enumerate}
  \def\labelenumi{(\arabic{enumi})}
  \setcounter{enumi}{164}
  \item
        Подумаешь так ― как все у нас кругом\ldots{} \textit{Грязь да грязь}.
        {[}А.~Терехов. Мемуары срочной службы (1991){]}
  \item
        А что отмечать? ― забурчали братья. ― \textit{Дождь да дождь}. ― Вот и
        отметьте условным знаком дождь и температуру проставьте. {[}Б.~Екимов.
        Фетисыч (1996){]}
\end{enumerate}

Конструкция выражает смысл: `нет ничего другого, кроме (снега, дождя,
грязи и~т.д.)'.

3.~(\textit{Ну}), \textit{медведь и медведь}. Ср.:

\begin{enumerate}
  \def\labelenumi{(\arabic{enumi})}
  \setcounter{enumi}{166}
  \item
        Он тебе никого не напоминает? ― \textit{Мужик и мужик},
        накачанный\ldots{} Лицо стандартное\ldots{} {[}М.~Баконина. Школа
        двойников (2000){]}
  \item
        Глянул Хрюк в корытце с водой ― и сам удивился: \textit{ну мальчик и
          мальчик}, только нос пятачком! {[}Б.~Заходер. Две сказочки
        (1960-1980){]}
\end{enumerate}

Конструкция означает, что называемый объект по своим свойствам является
типичным и заурядным представителем соответствующего класса объектов.
Ср.~странность сочетаний, подразумевающих исключительное, а не заурядное
проявление релевантного качества: \textsuperscript{?}\textit{Какой он
  накачанный! Мужик и мужик}, при допустимом: \textit{Какой он накачанный!
  Настоящий мужик}.

4.~(\textit{Ну}), \textit{упал и упал}. Ср.:

\begin{enumerate}
  \def\labelenumi{(\arabic{enumi})}
  \setcounter{enumi}{168}
  \item
        Ни о чём не говорили три падения Плющенко на петербургском этапе
        Гран-при ― \textit{упал и упал}, хотя и несколько странно. {[}«Известия»
        (2001){]}
  \item
        Сашку Ермолаева обидели. \textit{Ну}, \textit{обидели и обидели} ―
        случается. {[}В.~Шукшин. Обида (1970-1972){]}
\end{enumerate}

Конструкция выражает смысл: `то (чаще неприятное), что произошло,
происходит или может произойти, не имеет особого значения'. Данный смысл
не связан непосредственно с усилением, поэтому эта конструкция относится
к усилительной конъюнкции скорее в силу формального сходства.

Подробнее об этих и других сочинительных конструкциях с тождественными
словоформами см.~{[}Грамматика~1980{]}, {[}Крючкова~2004{]},
{[}Санников~2008{]}.

Подробнее о семантике соединительных союзов в русском языке см. статью
\underline{Сочинительные союзы~/ п.3.~Соединительные союзы}.

\subsubsection{Разделительность
  (дизъюнкция)}\label{ux440ux430ux437ux434ux435ux43bux438ux442ux435ux43bux44cux43dux43eux441ux442ux44c-ux434ux438ux437ux44aux44eux43dux43aux446ux438ux44f}

Семантика разделительности выражается такими союзами, как
рус.~\textit{или}, англ.~\textit{or}, нем.~\textit{oder} и~т.д.

В семантической зоне разделительности типологически центральным является
разграничение между обычной дизъюнкцией (171а) и альтернативным вопросом
(171б):

\begin{enumerate}
  \def\labelenumi{(\arabic{enumi})}
  \setcounter{enumi}{170}
  \item
        а.~Я позвоню сегодня или завтра. --- б.~Ты позвонишь сегодня или
        завтра?
\end{enumerate}

Различие между (а) и (б) состоит, в частности, в том, что в (б)
акцентирован выбор одной из альтернатив --- этот выбор находится в фокусе
вопроса; а в (а) обе альтернативы сообщены как возможные, но выбор между
ними не подчеркнут.

Различие между обычной дизъюнкцией и альтернативным вопросом не сводимо
к различию между утверждением и вопросом, поскольку обычная дизъюнкция
допустима и в контексте вопроса,
см.~\underline{Вопросительные~предложения}. Ср.~(курсивом выделены
слова, несущие коммуникативно значимые акценты):

\begin{enumerate}
  \def\labelenumi{(\arabic{enumi})}
  \setcounter{enumi}{171}
  \item
        а.~Ты \textit{хочешь} чай или кофе? --- б.~Ты хочешь \textit{чай} или
        \textit{кофе}?
\end{enumerate}

В (172а) обычная дизъюнкция помещена в общий вопрос, который
предполагает ответ \textit{да} или \textit{нет} и, тем самым, не
подчеркивает выбор между альтернативами. Напротив, ответ на
альтернативный вопрос (172б) предполагает выбор одной из альтернатив
(\textit{я хочу чай} или \textit{я хочу кофе}).

Во многих языках разграничение между обычной дизъюнкцией (в утверждениях
и вопросах) и альтернативным вопросом кодируется грамматическими
средствами. Так, в китайском и финском данные два вида дизъюнкции
оформляются разными союзами {[}Haspelmath~2007{]}. В русском языке
формальное различие проявляется в поведении повторяющего союза
\textit{или\ldots или} (см.~\underline{Сочинительные союзы~/ п.5.1.4}):
этот союз допустим при обычной дизъюнкции, но не в альтернативном
вопросе (в отличие от одиночного \textit{или}, ср.~(173в)). Ср.:

\begin{enumerate}
  \def\labelenumi{(\arabic{enumi})}
  \setcounter{enumi}{172}
  \item
        а.~Исполнять будут, вероятно, \textit{или классику}, \textit{или джаз}? --
        б.~*Исполнять будут \textit{или классику}, \textit{или джаз}? --
        в.~Исполнять будут \textit{классику или джаз}?
\end{enumerate}

Данная особенность \textit{или\ldots или} объясняется тем, что
альтернативный вопрос семантически распадается на несколько вопросов,
число которых определяется числом конъюнктов: \textit{Исполнять будут
  классику?} \textit{Или исполнять будут джаз?} В вопросе с
\textit{или\ldots или} наличие части союза перед первым конъюнктом
препятствует такому разложению: *\textit{Исполнять будут или классику?}
Аналогичным свойством обладают разделительные неодноместные союзы в
других языках европейского стандарта {[}Moravcsik~1971:~28{]}.

Предметом дискуссии является вопрос о том, выражают ли центральные
разделительные союзы (\textit{или}, \textit{or}, \textit{oder} и~проч.)
строгую (исключающую) или нестрогую (неисключающую) дизъюнкцию. Данный
вопрос обычно формулируется в логических терминах условий истинности.
При каких условиях истинно утверждение \textit{А или В}: при условии, что
истинно А или истинно В, но не А и В вместе (строгая дизъюнкция), или
при условии, что истинно А, или истинно В, или А и В вместе (нестрогая
дизъюнкция).

Согласно одной из точек зрения, \textit{или} может выступать в двух
значениях; выбор конкретного значения союза определяется контекстом
{[}Падучева~1964{]}, {[}Hurford~1974{]}, {[}Перетрухин~1979{]}. Ср.:

\begin{enumerate}
  \def\labelenumi{(\arabic{enumi})}
  \setcounter{enumi}{173}
  \item
        Иван Петрович умер \textit{во вторник или в среду} --- cтрогая
        дизъюнкция: невозможно умереть и во вторник, и в среду.
  \item
        Иван Петрович позвонит \textit{во вторник или в среду} --- нестрогая
        дизъюнкция; утверждение истинно в трех случаях: если Иван Петрович
        позвонит во вторник; если он позвонит в среду; если он позвонит и во
        вторник, и в среду
\end{enumerate}

Более убедительной, однако, представляется точка зрения,
сформулированная, в частности, в {[}Санников~2008:~48{]}, см.~также
{[}Haspelmath~2007{]}, согласно которой возможность совмещения ситуаций
А и В в значении союза \textit{или} вообще не учтена; такая возможность не
находится в фокусе внимания Говорящего. Поэтому союз \textit{или} уместен
и в тех контекстах, которые эту возможность допускают, и в тех, которые
ее исключают.

Против того, чтобы различать у \textit{или} два значения --- строгой и
нестрогой дизъюнкции --- говорит и тот факт, что ни в одном языке надежно
не засвидетельствовано разграничения двух значений формальными
средствами {[}Haspelmath~2007:~26{]}.

Подробнее о семантике разделительных союзов в русском языке см.~статью
\underline{Сочинительные союзы~/ п.5.~Разделительные союзы}.

\subsubsection{Противительность}\label{ux43fux440ux43eux442ux438ux432ux438ux442ux435ux43bux44cux43dux43eux441ux442ux44c}

Семантика противительности выражается такими союзами, как рус.
\textit{но}, \textit{а}, англ. \textit{but}, нем. \textit{aber} и~т.д.

В основе противительного значения лежит указание на обманутое ожидание
{[}Haspelmath~2007{]}, {[}Урысон~2011~и~др.{]}. Ср. предложение
\textit{Погода была дождливой}, \textit{но Петю повели гулять}: вторая
клауза опровергает ожидание, подсказанное первой клаузой (в дождливую
погоду не гуляют).

Семантическая зона противительности близка зоне уступительности;
последняя, однако, прототипически выражается подчинительными союзами,
ср. \textit{Хотя погода была дождливой}, \textit{Петю повели гулять}. По
мнению Е.~В.~Урысон, основное различие между союзами \textit{но} и
\textit{хотя} (и, шире, между противительностью и уступительностью)
состоит в том, что в семантике но центральным является само указание на
ожидание, тогда как в семантике хотя акцент --- на основании ожидания и
логике умозаключения, из этого основания проистекающего
{[}Урысон~2011:~203{]}, см.~также \underline{Подчинительные союзы~/
  п.6.~Уступительные союзы}.

Особенностью русского языка является наличие двух, одинаково центральных
противительных союзов --- \textit{но} и \textit{а}; у \textit{а} нет точного
эквивалента в основных европейских языках {[}Зализняк,~Микаэлян~2005{]}.
\textit{А} может выполнять, в частности, следующие функции, отсутствующие
у \textit{но}.

1.~Выражать контраст, не осложненный значением противоречия: \textit{У
  Коли мама физик}, \textit{а папа математик}; ср.~аналогичное предложение с
\textit{но}: \textit{У Коли мама физик}, \textit{но папа математик},
подразумевающее противоречие между описываемыми ситуациями.

2.~Выражать неосведомленность Говорящего о причинах того, почему
ожидание обманывается {[}Урысон~2011:~216{]}. Поэтому в случае, если
причина эксплицитно указана, союз \textit{а}, в отличие от \textit{но},
неуместен. Ср.:

\begin{enumerate}
  \def\labelenumi{(\arabic{enumi})}
  \setcounter{enumi}{175}
  \item
        Она не имела ни высшего, ни даже законченного среднего образования,
        \textit{но} \textless{}\textsuperscript{?}\textit{а}\textgreater{}
        благодаря удивительной интуиции была мудрее многих ученых
        специалистов. {[}Е.~Весник. Дарю, что помню (1997){]}
\end{enumerate}

В том же контексте, но без указания причины \textit{а} допустим: \textit{Она
  не имела ни высшего, ни даже законченного среднего образования, а была
  мудрее многих ученых специалистов}. Подробнее о семантике противительных
союзов в русском языке см.~статью \underline{Сочинительные союзы~/
  п.4.~Противительные союзы}.

\subsection{Границы сочинения как единой семантической
  зоны}\label{ux433ux440ux430ux43dux438ux446ux44b-ux441ux43eux447ux438ux43dux435ux43dux438ux44f-ux43aux430ux43a-ux435ux434ux438ux43dux43eux439-ux441ux435ux43cux430ux43dux442ux438ux447ux435ux441ux43aux43eux439-ux437ux43eux43dux44b}

Вопрос о том, можно ли говорить о сочинении как о единой семантической
зоне и если да, то каковы границы этой зоны, важен в следующем
отношении. Если бы существовала семантическая специфика сочинения, это
означало бы, что существуют такие семантические отношения, которые
заведомо не могут быть выражены сочинительным союзом; в этом случае
сочинение и подчинение могли бы быть разграничены на семантических
основаниях.

В действительности, однако, очертить четкие семантические границы
сочинения, по-видимому, невозможно. Можно говорить лишь о том, что
сочинение прототипически выражает семантически симметричные отношения --
соединительность, разделительность, сопоставление
(см.~\underline{п.3.~Симметрия сочинения}). Однако и такие отношения,
которым свойственна явная семантическая асимметрия, --- причина и
следствие, условие и следствие --- тоже не могут быть исключены из сферы
сочинения. Так, в основных европейских языках имеется сочинительный
причинный союз: англ.~\textit{for}, нем.~\textit{denn}, франц.~\textit{car}.

В русском языке имеются основания считать таким причинным сочинительным
союзом союз \textit{ибо} (от европейских аналогов \textit{ибо} отличается
книжной стилистикой). О сочинительности \textit{ибо} свидетельствуют его
синтаксические свойства --- канонические подчинительные причинные союзы
ведут себя противоположным образом (см.~\underline{Подчинительные
  союзы~/ п.2.~Причинные союзы}),

1.~Для клаузы, вводимой союзом \textit{ибо}, невозможна препозиция, ср.:

\begin{enumerate}
  \def\labelenumi{(\arabic{enumi})}
  \setcounter{enumi}{176}
  \item
        Мытьё обычно совершалось быстро, \textit{ибо}, согласно примете,
        девушка, которая скорее вымоется, и замуж быстрее выйдет. {[}«Народное
        творчество» (2004){]} --- *\textit{Ибо} \textless{}\textit{из-за того что},
        \textit{поскольку}\textgreater, согласно примете, девушка, которая
        скорее вымоется, и замуж быстрее выйдет, мытье обычно совершалось
        быстро.
\end{enumerate}

2.~\textit{Ибо} запрещает вынос вопросительного местоимения:

\begin{enumerate}
  \def\labelenumi{(\arabic{enumi})}
  \setcounter{enumi}{177}
  \item
        Кто выиграет, *ибо \textless{}\textit{если}, \textit{оттого
          что}\textgreater{} вы заболеете?
\end{enumerate}

3.~\textit{Ибо} не может вставляться в более крупную клаузу, требующую от
подчиненной клаузы какого-то формального изменения:

\begin{enumerate}
  \def\labelenumi{(\arabic{enumi})}
  \setcounter{enumi}{178}
  \item
        Я хочу, чтобы коммунисты проиграли, \textsuperscript{?}*\textit{ибо}
        \textless{}\textit{потому что}, \textit{из-за того что}, \textit{оттого
          что}\textgreater{} страна изменилась, а не из-за низкой явки.
\end{enumerate}

О формальных свойствах \textit{ибо} и других союзов см.~также
статью~\underline{Сочинение и подчинение}.

Следует отметить, что квалификация \textit{ибо} как сочинительного союза
противоречит традиционному делению союзов на сочинительные и
подчинительные, принятому, в частности, в АГ-80
{[}Грамматика~1980:~§3028{]}. Авторы АГ-80 относят к подчинительным все
союзы выраженно «асимметричной» семантики --- причинно-следственной и
условной-следственной (в частности: \textit{ибо}, \textit{так что}, \textit{а
  то}, \textit{иначе} и~т.п.)\footnote{Та же чисто семантическая
  классификация для удобства описания принимается в статьях
  \underline{Союз}, \underline{Сочинительные союзы},
  \underline{Подчинительные союзы}, поскольку данные статьи посвящены
  преимущественно именно семантике союзов. В частности в них описываются
  тонкие различия между отдельными союзами, принадлежащими к одной и той
  же семантической (но необязательно синтаксической) группе.}. Это
обусловлено лишь той общей предпосылкой, что сочинительные союзы
обслуживают семантически симметричные отношения, а подчинительные --
семантические асимметричные. Такая предпосылка, однако, опровергается
указанными выше типологическими фактами.

\section{Статистика}\label{ux441ux442ux430ux442ux438ux441ux442ux438ux43aux430}

Поскольку сочинение в русском языке прототипически выражается
сочинительными союзами, статистику, касающуюся сочинения, разумно
основывать на употреблении союзов. См. об этом статью
\underline{Сочинительные союзы}.

\section*{Библиография}\label{ux431ux438ux431ux43bux438ux43eux433ux440ux430ux444ux438ux44f}

\begin{itemize}
  \item
        Апресян Ю.Д. и др. Теоретические проблемы русского синтаксиса:
        Взаимодействие грамматики и словаря. М. 2010.
  \item
        Архипов А.В. Типология комитативных конструкций. М.: Знак. 2009.
  \item
        Белошапкова В.А. (Ред.) Современный русский язык. 2-е~изд. М.: Высшая
        школа. 1989.
  \item
        Белошапкова В.А. Сложное предложение в современном русском языке
        (некоторые вопросы теории). М.: Просвещение. 1967.
  \item
        Белошапкова В.А. Современный русский язык. Синтаксис. М.: Высшая
        школа. 1977.
  \item
        Богуславский И.М. Сфера действия лексических единиц. М. 1996.
  \item
        Вайс Д. Русские двойные глаголы и их соответствия в финно-угорских
        языках. Русский язык в научном освещении,~6(2). 2003.
  \item
        Вайс Д. Русские двойные глаголы: Кто хозяин, а кто слуга?~// Слово в
        тексте и словаре: Сборник статей к 70-летию акад. Ю.Д.~Апресяна. М.
        2000. С.~356--378.
  \item
        Гвоздев А.Н. Современный русский литературный язык. Синтаксис,~ч.II.
        М. 1973.
  \item
        Грамматика~1980 --- Шведова Н.Ю. (Ред.) Русская грамматика,~т.II. М.:
        Наука. 1980.
  \item
        Граудина Л.К., Ицкович В.А., Катлинская Л.П. Словарь грамматических
        вариантов русского языка. 3-е~изд., стер. М.: Астрель: АСТ. 2008.
  \item
        Зализняк А.А., Микаэлян И. Русский союз А как лингвоспецифическое
        слово~// Материалы конференции "Диалог 2005". М. 2005.
  \item
        Зализняк А.А., Падучева Е.В. К типологии относительного предложения~//
        Семиотика и информатика,~6. 1975 (переизд.: Семиотика и информатика.
        Opera Selecta,~35. М.: Языки русской культуры. 1997).
  \item
        Зализняк А.А., Падучева Е.В. О контекстной синонимии единственного и
        множественного числа существительных~// Информационные вопросы
        семиотики, лингвистики и автоматического перевода,~4. М. 1974.
        С.~30--35.
  \item
        Зализняк А.А., Падучева Е.В. Синтаксические свойства местоимения
        \textit{который}~//Категория определенности-неопределенности в
        славянских и балканских языках. М.: Наука. 1979 (Переиздано в
        сборнике: Падучева Е.В. Статьи разных лет. М.: Языки славянских
        культур. 2009. С.~86--116).
  \item
        Казенин К.И. Проблема "размеров" конъюнктов в русском языке: данные
        некоторых типов сочинительных конструкций // Вопросы языкознания,~4.
        2011. С.~46--61.
  \item
        Кибрик А.А., Подлесская В.И. (Ред.) Рассказы о сновидениях: Корпусное
        исследование устного русского дискурса. М.: ЯСК. 2009.
  \item
        Кодзасов С.В. Число в сочинительных конструкциях~// Кибрик А.Е.,
        Нариньяни А.С. (Ред.) Моделирование языковой деятельности в
        интеллектуальных системах. М.: «Наука». 1987. С.~201--219.
  \item
        Кодзасов С.В., Саввина Е.Н. Общие свойства сочинительных
        конструкций~// Кибрик А.Е., Нариньяни А.С. (Ред.) Моделирование
        языковой деятельности в интеллектуальных системах. М.: «Наука».
        1987.~С.~147--167.
  \item
        Крючкова О.Ю. Вопросы лингвистической трактовки лексической
        редупликации в русском языке~// Русский язык в научном
        освещении,~2(8). М. 2004. C.~63--85
  \item
        Лауфер Н.И. Линеаризация компонент сочинительной конструкции. В:
        Кибрик А.Е., Нариньяни А.С. (Ред.) Моделирование языковой деятельности
        в интеллектуальных системах. М.: Наука. 1987. С.~167--176.
  \item
        Падучева Е.В. Высказывание и его соотнесенность с действительностью.
        М. 1985.
  \item
        Падучева Е.В. О семантике синтаксиса. Материалы к трансформационной
        грамматике русского языка. М.: УРСС. 2007 (1-е изд. --- М.: Наука.
        1974).
  \item
        Перетрухин В.Н. Проблемы синтаксиса однородных членов предложения в
        современном русском языке. 1979.
  \item
        Пешковский А.М. Русский синтаксис в научном освещении. 8-е изд. М.
        2001 (1-е изд. --- М.: Гос. учебно-педагогическое изд-во Мин.
        просвещения РСФСР. 1928).
  \item
        Подлесская В.И. К типологии предикативного сочинения. Вопросы
        языкознания,~4. 1992.~С.~90--102.
  \item
        Подлесская В.И. Структурная и линейно-просодическая целостность
        именных групп по данным корпусного исследования: сочиненные и
        комитативные группы с личным местоимением первого лица в русском
        языке~// Вопросы языкознания,~1. 2012. С.~42--65.
  \item
        Подлесская В.И. Структурная и линейно-просодическая целостность
        именных групп по данным корпусного исследования: сочиненные и
        комитативные группы с личным местоимением первого лица в русском
        языке~// Вопросы языкознания,~1. 2012. С.~42--65.
  \item
        Прияткина А.Ф.~Русский язык: синтаксис осложненного предложения. М.
        1990.
  \item
        Санников В.З. Русские сочинительные конструкции. Семантика.
        Прагматика. Синтаксис. М.: Наука. 1989.
  \item
        Санников В.З. Русский синтаксис в семантико-прагматическом
        пространстве. М.: Языки славянских культур. 2008.
  \item
        Тестелец Я.Г. Введение в общий синтаксис. М. 2001.
  \item
        Урысон Е.В. Опыт описания семантики союзов. М. 2011.
  \item
        Ширяев Е.Н. Бессоюзное сложное предложение в современном русском
        языке. М.: Наука. 1986.
  \item
        Якобсон Р.О. В поисках сущности языка~// Семиотика. М.: Радуга. 1983.
        С.~102--117.
  \item
        Якобсон Р.О. Лингвистика и поэтика~// Структурализм: ``за'' и
        ``против''. М. 1975.
  \item
        Янко Т.Е. Коммуникативные стратегии русской речи. М. 2001.
  \item
        Blakemore D., Carston R. The pragmatics of sentential co-ordination
        with \textit{and}~// Lingua,~115(4). 2005. P.~569--589.
  \item
        Borsley R.D. Against ConjP. Lingua,~115. 2005. 461--482.
  \item
        Cooper W.E., Ross J.R. World order~// Grossman R.E.~et~al. (Eds.)
        Papers from the Parasession on Functionalism, Chicago Linguistic
        Society, University of Chicago. Chicago, Illinois. 1975. P.~63--111.
  \item
        Corbett G. Agreement. Cambridge: Cambridge University Press. 2006.
  \item
        Corbett G. Predicate Agreement in Russian (Birmingham Slavonic
        Monographs,~7). Birmingham: University of Birmingham. 1979.
  \item
        Cristofaro S. Deranking and balancing in different subordination
        relations: a typological study~// Sprachtypologie und
        Universalienforschung,~51. 1998.
  \item
        Cristofaro S. Subordination. Oxford University Press. 2003.
  \item
        Culicover P.W., Jackendoff R. 1997. Semantic subordination despite
        syntactic coordination. Linguistic Inquiry,~28. 1997. P.~195--218.
  \item
        Dik S.C. Coordination: its implications for a theory of general
        linguistics. Amsterdam. 1968.
  \item
        Grice H.P. Logic and conversation // SaS,~3. N.Y.: Acad. Press. 1975.
        P.~41--58.
  \item
        Haspelmath M. Coordination // Shopen~T. (Ed.) Language typology and
        syntactic description,~vol.2. Cambridge. 2007.
  \item
        Hurford J.R. Exclusive or Inclusive Disjunction // Foundations of
        Language,~11(3). 1974. P.~409--411.
  \item
        Johannessen J.B. Coordination. N.Y. --- Oxford: Oxford University
        Press. 1998.
  \item
        Lang E. The Semantics of Coordination. Amsterdam: Benjamins. 1984.
  \item
        Moravcsik E. On disjunctive connectives. Language Sciences,~15. 1971.
        P.~27--34.
  \item
        Munn A. First conjunct agreement: Against a clausal analysis~//
        Linguistic Inquiry,~75(3). 1999. P.~552--562.
  \item
        Munn A. Three types of coordination asymmetries~// Schwabe~K.,
        Zhang~N. (Eds.) Ellipsis in Conjunction. Tubingen: Max Niemeyer. 2000.
        P.~1--22.
  \item
        R.S. Kayne R.S. The Antisymmetry of Syntax. Cambridge. 1994.
  \item
        Ross J.R. Constraints on variables in syntax. PhD dissertation. MIT.
        1967 (Ross J.R. Infinite syntax! Ablex: Norwood. 1986).
  \item
        Sag I., Klein E., Wasow Th., Weisler S. 1985. Coordination and how to
        distinguish categories~// Natural Language and Linguistic Theory,~3.
        1985. P.~117--171.
  \item
        Van Oirsouw R. The syntax of coordination. L.: Croom Helm. 1987.
  \item
        te Velde J.R. Deriving Coordinate Symmetries. A Phase-based Approach
        integrating Select, Merge, Copy and Match. Amsterdam: Benjamins. 2005.
  \item
        Verstraete J.-Ch. Two types of coordination in clause combining~//
        Lingua,~115. 2005. P.~611--626.
  \item
        Wilder Ch. Coordination, ATB and ellipsis~// Zwart J.-W. (Ed.)
        Minimalism and Kayne's Antisymmetry Hypothesis. Groningen Arbeiten zur
        germanistischen Linguistik,~37. 1994. P.~291--329.
  \item
        Wilder Ch. Shared constituents and linearization~// Johnson K. (Ed.)
        Topics in Ellipsis, Cambridge: Cambridge University Press. 2008.
        P.~229--259.
  \item
        Wilder Ch. Some Properties of Ellipsis in Coordination~// Alexiadou
        A., Hall T.A. (Eds.) Studies on Universal Grammar and Typological
        Variation. Linguistik Aktuell~/~Linguistics Today,~13. Amsterdam:
        Benjamins. 1997. P.~59--107.
  \item
        Yuasa E., Sadock J.M. Pseudo-Subordination: A Mismatch Between Syntax
        and Semantics~// Journal of Linguistics,~38(1), 2002. P.~87--111.
\end{itemize}

\section*{Основная литература}\label{ux43eux441ux43dux43eux432ux43dux430ux44f-ux43bux438ux442ux435ux440ux430ux442ux443ux440ux430}

\begin{itemize}
  \item
        Белошапкова В.А. (Ред.) Современный русский язык. 2-е~изд. М.: Высшая
        школа. 1989. Часть "Синтаксис", глава~1.
  \item
        Кибрик А.Е., Нариньяни А.С. (Ред.) Моделирование языковой деятельности
        в интеллектуальных системах. М.: Наука. 1987. Глава~7.
  \item
        Пешковский А.М. Русский синтаксис в научном освещении. 8-е изд. М.
        2001 (1-е изд. --- М.: Гос. учебно-педагогическое изд-во Мин.
        просвещения РСФСР. 1928). Главы~5,~26,~27.
  \item
        Санников В.З. Русские сочинительные конструкции. Семантика.
        Прагматика. Синтаксис. М.: Наука. 1989.
  \item
        Тестелец Я.Г. 2001. Введение в общий синтаксис. М. 2001. Глава~4,
        раздел~6.
  \item
        Dik S.C. Coordination: its implications for a theory of general
        linguistics. Amsterdam. 1968.
  \item
        Haspelmath~M. (Ed.) 2004. Coordinating Constructions. Typological
        Studies in Language,~58. Amsterdam: John Benjamins.
  \item
        Haspelmath~M. The converb as a cross-linguistically valid category~//
        Martin Haspelmath~M., König~E. (Eds.) Converbs in Cross-linguistic
        Perspective. Berlin: Mouton de Gruyter. 1995. P.~1--55.
  \item
        Mithun M. 1988. The grammaticization of coordination~// Haiman~J.,
        Thompson~S.A. (Eds.) Clause Combining in Grammar and Discourse.
        Amsterdam: Benjamins. 1988. P.~31--60.
  \item
        R. Lakoff. If's, and's and but's about Conjunction~// Studies in
        Linguistic Semantics. N.Y. 1971.
  \item
        Stassen~L. AND-languages and WITH-languages~// Linguistic
        Typology,~4(1). 2000. P.~1--54.
  \item
        Van Oirsouw R.R. Coordination~// Joachim Jacobs J., von~Stechow~A.,
        Sternefeld~W., Venneman~Th. (Eds.) Syntax: An International Handbook
        of Contemporary Research,~vol.~1. Berlin: de Gruyter. 1993.
\end{itemize}
